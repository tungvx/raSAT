%\documentclass[conference, oribibl]{IEEEtran}
\documentclass[runningheads,a4paper,oribibl]{llncs}
%\usepackage{llncsdoc}

% *** MISC UTILITY PACKAGES ***
%
\usepackage{amssymb}
\usepackage{verbatim}
\setcounter{tocdepth}{3}
\usepackage{adjustbox,lipsum}
\usepackage{hyperref}
\usepackage{graphicx}
\graphicspath{ {Figs/} }

\usepackage{amsmath}
\usepackage{multirow}
%\usepackage{slashbox}
\usepackage{amsfonts}

\usepackage{algpseudocode}
\usepackage{algorithm}
\usepackage{epstopdf}
\usepackage{array}
\usepackage{enumerate}

\usepackage{epstopdf}

\usepackage{url}
%\urldef{\mailsa}\path|{khanhtv, mizuhito}@jaist.ac.jp|    

%ieee requirements
%\usepackage[utf8]{inputenc}
%\usepackage[T1]{fontenc}
%\usepackage{microtype} 
%\usepackage{balance}

%user definitions
\newcommand{\Nat}{{\mathbb N}}
\newcommand{\Real}{{\mathbb R}}
\newcommand{\Rat}{{\mathbb Q}}
\newcommand{\suppress}[1]{} % Comment out text.
\newcommand{\mizuhito}[1]{\{{\bf Mizuhito:~\sf #1}\}} % Highlight text.
\newcommand{\khanh}[1]{\{{\bf Khanh:~\sf #1}\}} % Highlight text.

\newcommand{\smallHead}[1]{%
    \par\vspace{.35cm}\noindent\textbf{#1}%
    \par\noindent\ignorespaces%
}

\newcommand\TTTT{%
 \textsf{T\kern-0.2em\raisebox{-0.3em}T\kern-0.2emT\kern-0.2em\raisebox{-0.3em}2}%
}

% correct bad hyphenation here
\hyphenation{op-tical net-works semi-conduc-tor}


\begin{document}
%
% paper title
% can use linebreaks \\ within to get better formatting as desired
% Do not put math or special symbols in the title.
\title{{\bf raSAT 0.2} for SMT-COMP 2015}

\author{Vu Xuan Tung\inst{1}, To Van Khanh\inst{2}, and Mizuhito Ogawa\inst{1}} 
\institute{
Japan Advanced Institute of Science and Technology\\
\email{\{tungvx,mizuhito\}@jaist.ac.jp}
\and 
University of Engineering and Technology, Vietnam National 
University, Hanoi \\
\email{khanhtv@vnu.edu.vn}
}

%\tableofcontents

% make the title area
\maketitle

% As a general rule, do not put math, special symbols or citations
% in the abstract
{\bf raSAT} is an SMT solver for polynomial constraints. 
It consists of a simple iterative approximation refinement, called {\bf raSAT} loop~\cite{VanKhanh201227}, 
which is an extension of the standard ICP (Interval Constraint Propagation) with Testing. 
Two approximation schemes consist of Interval Arithmetic (IA) and 
Testing, to accelerate SAT detection. 
If both fails, input intervals are refined by decomposition.  

raSAT loop is extended with the use of the Intermediate Value Theorem to show the satisfiability of equations. 

To avoid soundless bugs due to round-off error of floating arithmetic operations, \textbf{raSAT} applies outward rounding in Interval Arithmetic and implements SAT confirmation step by an error-bound guaranteed floating point package {\bf iRRAM}\footnote{% 
\tt http://irram.uni-trier.de}. 

\textbf{raSAT} takes advantages from the following packages/libraries.
\begin{itemize}
\item {\bf miniSAT}\footnote{http://minisat.se/} as the back-end SAT solver.
\item {\bf iRRAM} for confirmation of SAT instances.
\item The library in \cite{Al2012.14} for
round-down/up in each Interval Arithmetics.
\item The OCaml parser for SMT-LIB 2.0 scripts from \url{http://smtlib.cs.uiowa.edu/utilities.shtml}.
\end{itemize}

\section*{Package Distribution:} Source code and a precompiled version of \textbf{raSAT} can be downloaded from \url{http://www.jaist.ac.jp/~s1310007/raSAT/}.



%\balance
\bibliographystyle{splncs03}
\bibliography{generic}


\end{document}
