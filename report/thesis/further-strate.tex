\chapter{Further Strategies}
\section{UNSAT Core}
In \tiny IA\_UNSAT \normalsize rule, the negation of $\mathring\Pi$ is added into the interval constraint so that $\mathring\Pi$ will not be explore again later because it make the constraint unsatisfiable. If we can find $\mathring\Pi'$ such that $\mathring\Pi = \mathring\Pi' \wedge \mathring\Pi''$ and $\varphi$ is $\{\mathring\Pi'^p_{IA}\}$-UNSAT, we can add $\neg\mathring\Pi'$ into interval constraint instead of $\mathring\Pi$ to reduce the search space.

\begin{example}
Consider the constraint $\varphi = x^2 + y^2 < 1$. Suppose in the \tiny IA\_UNSAT \normalsize rule, we have $\Pi = (x \in \langle 2, 3 \rangle \vee x \in \langle 0, 2 \rangle) \wedge (y \in \langle 0, 1 \rangle y \in \langle -1, 0 \rangle)$ and $\mathring\Pi = x \in \langle 2, 3 \rangle \wedge y \in \langle 0, 1 \rangle$. The conditions of \tiny IA\_UNSAT \normalsize are satisfied, $\neg \mathring\Pi$ is added into the interval constraint which becomes $\Pi \wedge \neg \mathring\Pi$. The new interval constraint contains $\{x \in \langle 2, 3 \rangle, y \in \langle -1, 0 \rangle\}$ as one of its solution. However, with $\mathring\Pi' = x \in \langle 2, 3 \rangle$, we have $\varphi$ is $\{\mathring\Pi'^p_{IA}\}$-UNSAT and by adding $\neg\mathring\Pi'$ to the interval constraint, $\{x \in \langle 2, 3 \rangle, y \in \langle -1, 0 \rangle\}$ is also removed from the search space.
\end{example}

The constraint $\varphi = \bigwedge\limits_{j=1}^n f_j > 0$ is $\mathring\Pi$-UNSAT when $f_k > 0$ is $\mathring\Pi$-UNSAT with some $k \in \{1, 2, \cdots, n\}$. We have two ideas for computing UNSAT core.
\begin{enumerate}
\item \emph{UNSAT core 1:} A sub-polynomial $f_k'$ of $f_k$ such that
\begin{itemize}
\item[$\bullet$] $f_k' > 0$ is $\mathring\Pi$-UNSAT implies that $f_k$ is $\mathring\Pi$-UNSAT, and
\item[$\bullet$] $f_k'$ is in fact $\mathring\Pi$-UNSAT.
\end{itemize} 
In this case, we just take $\mathring\Pi' = \bigwedge\limits_{v_i \in var(f_k)}v_i \in \langle l_i, h_j \rangle$.
\item \emph{UNSAT core 2:} Check all the possible cases if $\mathring\Pi'$.
\end{enumerate}

\begin{example}
Consider again the constraint $\varphi = x^2 + y^2 < 1$ or $1 - x^2 - y^2 > 0$. In the \tiny IA\_UNSAT \normalsize rule, we also have $\Pi = (x \in \langle 2, 3 \rangle \vee x \in \langle 0, 2 \rangle) \wedge (y \in \langle 0, 1 \rangle y \in \langle -1, 0 \rangle)$ and $\mathring\Pi = x \in \langle 2, 3 \rangle \wedge y \in \langle 0, 1 \rangle$. Here, $\varphi$ is $\Pi$-UNSAT. In addition, $1 - x^2$ is the UNSAT core of $1 - x^2 - y^2$ because 
\begin{itemize}
\item[$\bullet$] $1 - x^2$ is $\Pi^p_{IA}$-UNSAT implies that $1 - x^2 - y^2$ is $\Pi^p_{IA}$-UNSAT, and 
\item[$\bullet$] the constraint $1 - x^2 > 0$ is in fact $\Pi^p_{IA}$-UNSAT.
\end{itemize} 
\end{example}

\subsection*{Preliminary Experiments}
Strategy \emph{UNSAT core 1} was already implemented in \cite{VanKhanh201227} and the other strategy was implemented in this work. We did the experiments on the SMT-LIB family Hong where each problem is unsatisfiable and of the following form:
\[\sum_{i=1}^{n}x_i^2 < 1 \wedge \prod_{i=1}^{n}x_i > 1\]
where $n$ ranges from $1$ to $20$. Table~\ref{tab:hong} shows the experiments for raSAT with/without UNSAT core strategies where we fixed the initial intervals to be $\langle 0, 10 \rangle$ and the threshold $0.1$ (i.e. no incremental deepening and widening). \emph{UNSAT core 2} works fine because problems in Hong family contain polynomials with lots of UNSAT cores which this strategy tends to calculate. On the other hand, \emph{UNSAT core 1} has not been well implemented in the way raSAT search for a sub-polynomial. Experiments for these strategies on other benchmarks of SMT-LIB did not show improvements in the result, we need further investigations.

\begin{table}
\begin{center}
\begin{tabular}{| c | c | c | c | c |  c | c |}
\hline
\multirow{2}{*}{Problem} & \multicolumn{2}{ c }{No UNSAT core} & \multicolumn{2}{ |c }{UNSAT core 1} & \multicolumn{2}{ |c| }{UNSAT core 2}\\ \cline{2-7}& Time (s) & Result & Time (s) & Result& Time (s) & Result\\ \hline
hong\_1 & 0 & UNSAT & 0 & UNSAT& 0.004 & UNSAT\\ \hline
hong\_2 & 0.00838 & UNSAT & 0.016 & UNSAT & 0.016 & UNSAT\\ \hline
hong\_3 & 0.007441 & UNSAT & 0.016 & UNSAT & 0.016 & UNSAT\\ \hline
hong\_4 & 0.114857 & UNSAT & 0.124 & UNSAT & 0.016 & UNSAT\\ \hline
hong\_5 & 0.27588 & UNSAT & 0.272 & UNSAT & 0.028 & UNSAT\\ \hline
hong\_6 & 1.20687 & UNSAT & 1.288 & UNSAT & 0.052 & UNSAT\\ \hline
hong\_7 & 9.29289 & UNSAT & 9.996 & UNSAT & 0.112 & UNSAT\\ \hline
hong\_8 & 153.619 & UNSAT & 164.288 & UNSAT& 0.68 & UNSAT\\ \hline
hong\_9 & 117.937 & UNSAT & 129.044 & UNSAT& 0.08 & UNSAT\\ \hline
hong\_10 & 307.208 & UNSAT & 281.696 & UNSAT& 0.152 & UNSAT\\ \hline
hong\_11 & 478.605 & UNSAT & 412.028 & UNSAT & 0.236 & UNSAT\\ \hline
hong\_12 & 500 & Timeout & 500 & Timeout & 0.456 & UNSAT\\ \hline
hong\_13 & 500 & Timeout & 500 & Timeout & 0.752 & UNSAT\\ \hline
hong\_14 & 500 & Timeout & 500 & Timeout & 1.572 & UNSAT\\ \hline
hong\_15 & 500 & Timeout & 500 & Timeout & 2.756 & UNSAT\\ \hline
hong\_16 & 500 & Timeout & 500 & Timeout & 5.98 & UNSAT\\ \hline
hong\_17 & 500 & Timeout & 500 & Timeout & 10.864 & UNSAT\\ \hline
hong\_18 & 500 & Timeout & 500 & Timeout & 24.352 & UNSAT\\ \hline
hong\_19 & 500 & Timeout & 500 & Timeout & 47.968 & UNSAT\\ \hline
hong\_20 & 500 & Timeout & 500 & Timeout & 103.484 & UNSAT\\ \hline
\end{tabular}
\caption{Experiments on UNSAT core computations}
\label{tab:hong}
\end{center}
\end{table}


\section{Test Case Generation} \label{sec:testGen}
\sloppy
The value of variable's sensitivity can also be used to approximate how likely the value of a polynomial increases when the value of that variable increases. Consider the constraint $f = -x_{15}*x_8+x_{15}*x_2-x_{10}*x_{16}>0$. With ${x_2 \in [9.9, 10]}, {x_8 \in [0, 0.1]}, {x_{10} \in [0, 0.1]}, {x_{15} \in [0, 10]},$ and $ x_{16} \in [0, 10]$. The result of AF2 for $f$ is: $0.25 \epsilon_2 - 0.25 \epsilon_8 - 0.25 \epsilon_{10} + 49.5\epsilon_{15} - 0.25\epsilon_{16} + 0.75\epsilon_{+-} + 49.25$. The coefficient of $\epsilon_2$ is positive ($0.25$), then we expect that if $x_2$ increases, the value of $f$ also increase.  As the result, the test case of $x_2$ is as high as possible in order to satisfy $f>0$. We will thus take the upper bound value of $x_2$, i.e. $10$. Similarly, we take the test cases for other variables: $x_8=0, x_{10}=0, x_{15}=10, x_{16}=0$. With these test cases, we will have $f=100 > 0$.
\subsection*{Preliminary Experiments}
Table~\ref{tab:senInTest} illustrates the experiments of this strategy together with (1)-(5)-(8). In comparison with (1)-(5)-(8), this strategy solves more satisfiable constraints and the solving time is generally smaller for the same constraint.
\begin{table} \label{tab:senInTest}
\begin{center}
\begin{tabular}{| c | c | c | c | c |}
\hline
Benchmark & \multicolumn{2}{|c|}{SAT} & \multicolumn{2}{c|}{UNSAT}\\ \hline
Zankl/matrix-1 (53) & 24 & 511.07 (s) & 2 & 0.009(s) \\ \hline
Zankl/matrix-2,3,4,5 (98) & 13 & 477.62 (s) & 8 & 0.39(s) \\ \hline
\end{tabular}
\end{center}
\caption{raSAT with sensitivity in testing}
\end{table}
\section{Box Decomposition}
Currently raSAT applies balanced decomposition in \tiny REFINE \normalsize rule, e.g. ${x \in \langle 0, 10 \rangle}$ will be decomposed into ${x \in \langle 0, 5 \rangle}$ and ${x \in \langle 5, 10 \rangle}$. We intend to use the same approximation as in Section~\ref{sec:testGen} to guide the decomposition. Take the same example in Section~\ref{sec:testGen} and suppose $x_{15}$ is selected for decomposition. Because the coefficient of $\epsilon_{15}$ is 49.5 which is positive, we expect the high values for $x_{15}$ so that $f > 0$ will be satisfied. As the result, the interval ${x_{15} \in \langle 0, 10 \rangle}$ can be decomposed into ${x_{15} \in \langle 0, 10 - \epsilon \rangle}$ and ${x_{15} \in \langle 10 - \epsilon, 10 \rangle}$ where $\epsilon$ is the threshold for decomposition.

\subsection*{Preliminary Experiments}
Table~\ref{tab:senInDec} shows the result when this strategy is used together with (1)-(5)-(8) and the strategy in Section~\ref{sec:testGen}. Basically the result has not been improved in comparison with the result in Section~\ref{sec:testGen}, we need further investigation.
\begin{table} \label{tab:senInDec}
\begin{center}
\begin{tabular}{| c | c | c | c | c |}
\hline
Benchmark & \multicolumn{2}{|c|}{SAT} & \multicolumn{2}{c|}{UNSAT}\\ \hline
Zankl/matrix-1 (53) & 24 & 510.55 (s) & 2 & 0.01(s) \\ \hline
Zankl/matrix-2,3,4,5 (98) & 9 & 1030.35 (s) & 8 & 0.38(s) \\ \hline
\end{tabular}
\end{center}
\caption{raSAT with sensitivity in decomposition}
\end{table}


% % % % % % % % % % % % % %