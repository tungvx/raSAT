\chapter{Equality handling}
\subsection{Greater-than-or-Equal Handling} \label{sec:geq}

{\bf raSAT} loop is designed to solve polynomial inequality. 
There are several ways to extend to handle equality, in which our idea shares similarity with 
dReal~\cite{dRealCADE13,dRealLICS12}. 

\begin{definition} \label{def:strict_unsat}
$\bigwedge \limits_{j} f_j \geq 0$ is 
{\em strict-SAT} (resp. {\em strict-UNSAT}) 
if $\bigwedge \limits_{j} f_j > \delta_j$ is SAT 
(resp. $\bigwedge \limits_{j} f_j > -\delta_j$ is UNSAT) for some $\delta_j >0$.
\end{definition}

\begin{lemma} \label{lem:strict_sat}
If $\bigwedge \limits_{j} f_j \geq 0$ is {\em strict-SAT} (resp. {\em strict-UNSAT}), 
it is SAT (resp. UNSAT).
\end{lemma}

Note that netiher strict-SAT nor strict-UNSAT (i.e., kissing situation), 
%if $\bigwedge \limits_{j} f_j \geq 0$ is SAT but $\bigwedge \limits_{j} f_j > 0$ is UNSAT 
Lemma~\ref{lem:strict_sat} cannot conclude anything, and \textbf{raSAT} says {\em unknown}. 
%In implementation of \textbf{raSAT}, when $\geq$ appears, exploration of IA-SAT 
%(resp. IA-UNSAT) is reduced to that of IA-strict-SAT (resp. IA-strict-UNSAT). 


\subsection{SAT on Equality by Intermediate Value Theorem} \label{sec:eq}
For solving polynomial constraints with single equality ($g=0$), we apply {\em Intermediate Value Theorem}. 
That is, if existing 2 test cases such that $g > 0$ and $g < 0$, then $g=0$ is SAT somewhere in between, 
as in Fig.~\ref{fig:ivt}. 

\begin{lemma} \label{lemma:ivt}
For $F = \exists x_1 \in (a_1,b_1) \wedge \cdots \wedge x_n \in (a_n,b_n). 
\bigwedge \limits_{j}^m f_j > 0~\wedge~g = 0$, $F$ is SAT, if 
there is a box $(l_1, h_1) \times \cdots \times (l_n,h_n)$ with $ (l_i,h_i) \subseteq (a_i,b_i)$ 
such that 
\begin{enumerate}[(i)]
\item $\bigwedge \limits_{j}^m f_j > 0$ is IA-VALID in the box, and 
\item there are two instances $\vec{t},\vec{t'}$ in the box with $g(\vec{t}) > 0$ and $g(\vec{t'}) < 0$.
\end{enumerate}
\end{lemma}

{\bf raSAT} first tries to find an IA-VALID box for $\bigwedge \limits_{j}^m f_j > 0$ by refinements. 
If such a box is found, it tries to find 2 instances for $g > 0$ and $g < 0$ by testing. 
Intermediate Value Theorem guarantees the existence of an SAT instance in between. 
Note that this method works for single equality and does not find an exact SAT instance. 
If multiple equalities do not share variables each other, we can apply Intermediate Value Theorem 
repeatedly to decide SAT. In Zankl benchmarks in SMT-lib, there are 15 gen-**.smt2 that contain equality
(among 166 problems), and each of them satisty this condition. 


 
\suppress{
In Table \ref{tab:eqexp} we show preliminary experiment for 15 problems that contain polynomial equalities in Zankl family. \textbf{raSAT} works well for these SAT problems and it can detect all SAT problems (11 among 15). At the current implementation, raSAT reports \emph{unknown} for UNSAT problems. The first 4 columns indicate \emph{name of problems}, \emph{the number of variables}, \emph{the number of polynomial equalities} and \emph{the number of inequalities}  in each problem, respectively. The last 2 columns show comparison results of \textbf{Z3 4.3} and \textbf{raSAT}.
\begin{table}
\centering
\scalebox{1.0}{
\begin{tabular}[b]{|c|c|c|c|c|c|c|c|}
\hline
%\multirow{2}{*}{Problem} & {No.} & {No.} & {No.}&
{Problem} & {No.} & {No.} & {No.}&
\multicolumn{2}{c|}{\textbf{Z3 4.3} (15/15)} &\multicolumn{2}{c|}{\textbf{raSAT} (11/15)}\\
\cline{5-8}
Name & Variables& Equalities& Inequalities&{Result} & {Time(s)}&{Result} & {Time(s)}\\
\hline
gen-03 & 1 & 1 & 0& SAT &0.01 & SAT &0.015\\
\hline
gen-04 & 1 & 1 & 0& SAT &0.01 & SAT &0.015\\
\hline
gen-05 & 2 & 2 & 0& SAT &0.01 & SAT &0.046\\
\hline
gen-06 & 2 & 2 & 1& SAT &0.01 & SAT &0.062\\
\hline
gen-07 & 2 & 2 & 0& SAT &0.01 & SAT &0.062\\
\hline
gen-08 & 2 & 2 & 1& SAT &0.01 & SAT &0.062\\
\hline
gen-09 & 2 & 2 & 1& SAT &0.03 & SAT &0.062\\
\hline
gen-10 & 1 & 1 & 0& SAT &0.02 & SAT &0.031\\
\hline
gen-13 & 1 & 1 & 0& UNSAT &0.05 & unknown &0.015\\
\hline
gen-14 & 1 & 1 & 0& UNSAT &0.01 & unknown &0.015\\
\hline
gen-15 & 2 & 3 & 0& UNSAT &0.01 & unknown &0.015\\
\hline
gen-16 & 2 & 2 & 1& SAT &0.01 & SAT &0.062\\
\hline
gen-17 & 2 & 3 & 0& UNSAT &0.01 & unknown &0.031\\
\hline
gen-18 & 2 & 2 & 1& SAT &0.01 & SAT &0.078\\
\hline
gen-19 & 2 & 2 & 1& SAT &0.05 & SAT &0.046\\
\hline
\end{tabular}
}
\caption{Experimental results for 15 equality problems of Zankl family}
\label{tab:eqexp}
\end{table}

We also apply the same idea for multiple equalities $\bigwedge \limits_{i} g_i = 0$ such that $Var(g_k) \cap Var(g_{k'}) = \emptyset$ where $Var(g_k)$ is denoted for the set of variables in the polynomial $g_k$. In the next section we will present idea for solving general cases of multiple equalities.
}

}
%%%%%%%%%%%%%%%%%%%%%%%%%%%%%
