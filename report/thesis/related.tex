\chapter{Related Works} \label{chap:related}
\section{Methodologies for Polynomial Constraints over Real Numbers}
Although solving polynomial constraints on real numbers is decidable~\cite{tarski}, current methodologies have their own pros and cos. They can be classified into the following categories: 
\begin{enumerate}
\item \textbf{Quantifier Elimination by Cylindrical Algebraic Decomposition (QE-CAD)}~\cite{qecad} 
is a complete technique, and 
is implemented in Mathematica, Maple/SynRac, Reduce/Redlog, QEPCAD-B, and recently 
in
Z3 4.3 (which is referred as nlsat in~\cite{Jovanovic13}).
Although QE-CAD is precise and detects beyond SAT instances (e.g., SAT regions), 
scalability is still challenging, since its complexity is doubly-exponential with respect to the number of variables. 
%Since QE-CAD is DEXPTIME wrt the number of variables, 

\item \textbf{Virtual Substitution } eliminates an existential quantifier by substituting the corresponding quantified variable with a very small value ($-\infty$), and either each root (with respect to that variable) of polynomials appearing in the constraint or each root plus an infinitesimal $\epsilon$. Disjunction of constraints after substitutions is equivalent to the original constraint. Because VS needs the formula for roots of polynomials, its application is restricted to polynomials of degree up to 4. SMT-RAT and  
Z3 \cite{PBM12} applies VS.

\item \textbf{Bit-blasting}. 
In this category of methodology, numerical variables are represented by a sequence of binary variables. The given constraint is converted into another constraint over the boolean variables. SAT solver is then used to find a satisfiable instance of binary variables which can be used to calculate the values of numerical variables.  MiniSmt~\cite{Zankl:2010:SNR:1939141.1939168}, the winner of QF\_NRA in SMT competition 2010, 
applies it for (ir)-rational numbers.
It can show SAT quickly, but due to the bounded bit encoding, 
it cannot conclude UNSAT. In addition, high degree of polynomial results in large SAT formula which is an obstacle of bit-blasting.

\item \textbf{Linearization}. ~
CORD \cite{cordic} uses COrdinate Rotation DIgital Computer (CORDIC) for real numbers to linearizes multiplications into a sequence of linear constraints. Each time one multiplication is linearized, a number of new constraints and new variables are introduced. As a consequence, high degree polynomials in the original constraint lead to large number of linear constraints. 

\item \textbf{Interval Constraint Propagation (ICP)} 
which are used in SMT solver community, e.g., iSAT3~\cite{isat}, 
dReal~\cite{dRealCADE13}, and RSOLVER~\cite{rsolver}. 
ICP combines over-approximation by interval arithmetics and constraint propagation to prune out the set of unsatisfiable points. When pruning does not work, decomposition (branching) on intervals is applied. 
ICP which is capable of solving "multiple thousand arithmetic constraints over some thousands of variables" \cite{isat} is practically often more efficient than algebraic computation.
\end{enumerate}

Because \textbf{raSAT} in the same category with \text{iSAT3} and \text{dReal}, next section is going to take a look at details of methodologies used in these solvers.

\section{Solvers using Interval Constraint Propagation}
\subsection*{iSAT3}
Although \textbf{iSAT3} also uses Interval Arithmetic, its algorithm integrates IA with DPLL procedure tighter than one of \textbf{raSAT}. During DPLL procedure, in addition to an assignment of literals, \textbf{iSAT3} also prepares a data structure to store interval boxes where each box corresponds to one decision level of DPLL procedure's assignment. In \textbf{UnitPropagation} rule, intead of using standard rule, \textbf{iSAT3} searches for clauses that have all but one atoms being inconsistence with the current interval box. When some atom are selected for the literals assignment, this tool tries to use the selected atoms to contract the corresponding box to make it smaller. In order to do this, \textbf{iSAT3} convert each inequality/equation in the given constraints into the conjunction of the atoms of the following form by introducing additional variables:
\begin{center}
\begin{tabular}{l c l}
atom &::=& bound $\mid$ equation \\
bound &::=& variable relation rational\_constant \\
relation &::=& $< \mid \le \mid = \mid \ge \mid >$ \\
equation &::=& variable = variable bop variable \\
bop &::=& $+ \mid - \mid \times$
\end{tabular}
\end{center}
In other words, the resulting atoms are of the form, e.g., either $x > 10$ or $x = y + z$. For example, the constraint \[x^2 + y^2 < 1\]
is converted into:
\[\left\{ 
  \begin{array}{l}
    x_1 = x^2\\
    x_2 = y^2\\
    x_3 = x_1 + x_2\\
    x_3 < 1
   \end{array}
    \right.\]
From the atoms of these form, the contraction can be easily done for interval boxes:
\begin{itemize}
\item[$\bullet$] For the bound atom of the form, e.g., $x > 10$, if the bound is $x \in \langle 0, 100 \rangle$, then the contracted box contain $x \in \langle 10, 100 \rangle$.
\item[$\bullet$] For the equations of three variables $x = y \text{ bop } z$, from bounds of any two variables, we can infer the bound for the remaining one. For example, from
\[\left\{ 
  \begin{array}{l}
    x = y.z\\
    x \in \langle 1, 10 \rangle \\
    y \in \langle 3, 7 \rangle\\
   \end{array}
    \right.\]
we can infer that \[z \in \langle \frac{1}{7}, \frac{10}{3} \rangle \]
\end{itemize}
When the \textbf{UnitPropagation} and contraction can not be done, \textbf{iSAT3} split one interval (decomposition) in the current box and select one decomposed interval to explore which corresponds to \textbf{decide} step. If the contraction yields an empty box, a conflict is detected and the complement of the bound selection in the last split needs to be asserted. This is done via \textbf{learn} the causes of the conflict and \textbf{backjump} to the previous bound selection of the last bound selection. In order to reason about causes of a conflict, \textbf{iSAT3} maintains an implication graph to represents which atoms lead to the asserting of one atom. 
\subsection*{dReal}
In stead of showing satisfiability/unsatisfiability of the polynomial constraints $\varphi$ over the real, \textbf{dReal} solves the follow problem given $\varphi$:


% % % % % % % % % % % % % %