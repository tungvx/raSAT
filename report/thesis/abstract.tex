\begin{abstract}
This thesis presents an SMT solver {\bf raSAT} for polynomial constraints. Solving polynomial constraints is raised from many applications of Software verification such as roundoff/overflow error analysis, automatic termination proving or loop invariant generation. In 1951, Tarski proposed a decision method for solving polynomial constraints over real numbers.  
{\bf raSAT} consists of a simple iterative approximation refinement, called {\bf raSAT} loop, 
which is an extension of the standard ICP (Interval Constraint Propagation) with testing. 
Two approximation schemes consist of interval arithmetic (over-approximation) and 
testing (under-approximation), to accelerate SAT detection. 
If both fails, input intervals are refined by decomposition. 

ICP is robust for large degrees, but the number of boxes (products of intervals) to explore 
exponentially explodes when the number of variables increases. 
We design strategies for boosting SAT detection on the choice of a variable to decompose
and a box to explore. 

to choose a variable for decomposition and a box by 
targeting on SAT detection. 

Several heuristic measures, called {\em SAT likelyhood}, {\em sensitivity}, and the number of 
unsolved atomic polynomial constraints, are compared on Zankl and Meti-Tarski benchmarks from 
QF\_NRA category of SMT-LIB. They are also evaluated by comparing {\bf Z3 4.3} and {\bf iSAT3}. 
We also show a simple modification to handle mixed intergers, and experiments on 
AProVE benchmark from QF\_NIA category of SMT-LIB.
\end{abstract}