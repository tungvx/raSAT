\chapter*{\centering Abstract} 
Solving polynomial constraints is raised from many applications of Software Verification such as roundoff/overflow error analysis, automatic termination proving or loop invariant generation. Although in 1948, Tarski proved the decidability of polynomial constraints over real numbers, the current complete method named Quantifier Elimination by Cylindrical Algebraic Decomposition has the complexity of doubly-exponential with respect to the number of variables which remains as an impediment. Interval Constraint Propagation (ICP) which use the inequalities/equations to contract the interval of variables by removing the unsatisfiable intervals is an efficient methodology because it uses floating point arithmetic. However the number of boxes (combination of intervals of variables) may grow exponentially.

This thesis presents strategies for efficiency improvement and extensions of an SMT solver named {\bf raSAT} for polynomial constraints. {\bf raSAT} which initially focuses on polynomial inequalities over real numbers follows ICP methodology and adds testing to boost satisfiability detection. In this work, in order to deal with exponential exploration of boxes, several heuristic measures, namely {\em SAT likelyhood}, {\em sensitivity}, and \emph{the number of 
unsolved polynomial inequalities}, are proposed. From the experiments on standard SMT-LIB benchmarks, \textbf{raSAT} is able to solve large constraints (in terms of the number of variables) which are difficult for other tools. 
In addition to those heuristics, extensions for handling equations using the Intermediate Value Theorem and handling constraints over integer number are also presented in this thesis.