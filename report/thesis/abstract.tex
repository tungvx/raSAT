\begin{abstract}
This thesis presents strategies for efficiency improvement and extensions of an SMT solver {\bf raSAT} for polynomial constraints. 
\begin{comment}
Solving polynomial constraints is raised from many applications of Software Verification such as roundoff/overflow error analysis, automatic termination proving or loop invariant generation. In 1948, Tarski proved the decidability of polynomial constraints over real numbers by proposing a decision method \cite{tarski} which is, however, "totally impractical" \cite{Davenport198829}. Later in \citeyear{Collins:1976:QER:1093390.1093393} \citet{Collins:1976:QER:1093390.1093393} introduced an algorithm called Cylindrical Algebraic Decomposition (CAD) which is complete but its time complexity is doubly-exponential with respect to the number of variables. A number of incomplete methodologies have been also proposed. MiniSmt employs bit-blasting method which is "dedicated to satisfiable instances only" \cite{Zankl:2010:SNR:1939141.1939168}. Virtual Substitution which is implemented in SMT-RAT \cite{smtrat} and Z3 \cite{PBM12} requires polynomials with degree less than $5$. CORD \cite{cordic} linearizes each multiplication as a sequence of linear constraint (with new variables) which makes the size of the problem become large when high degree polynomials present. Another approach is Interval Constraint Propagation (ICP) which use the inequalities/equations to contract the interval of variables by removing the unsatisfiable intervals. ICP uses floating point arithmetic so it does not suffers from high degree of polynomial, bt the number of boxes (combination of intervals of variables) may grow exponentially. As a result, ICP-based solvers need to have strategies to overcome this drawback.
\end{comment}

{\bf raSAT} which initially focuses on polynomial inequalities over real numbers follows ICP methodology and adds testing to boost satisfiability detection \cite{VanKhanh201227}. In this work, in order to deal with exponential exploration of boxes, several heuristic measures, called {\em SAT likelyhood}, {\em sensitivity}, and the number of 
unsolved atomic polynomial constraints, are proposed. Extensions for handling equations and handling constraints over integer number are also presented.
\end{abstract}