\chapter{Variation of Interval Arithmetic}
Interval Arithmetic is defined formally in Section ~\ref{sec:IA} of Chapter ~\ref{chap:OT-UT}. This chapter is going to present two instances of Interval Arithmetic which are used in raSAT: Classical Interval and Affine Interval. These two kinds differ to each other in the way they represent intervals and interpret function symbols.

\section{Classical Interval}
A model $M^p_{CI} = (U^p_{CI}, I^p_{CI})$ over intervals contains a set of all intervals $U^p_{CI} = U^p_{IA}$ and a map $I^p_{CI}$ that satisfies the following conditions.
\begin{enumerate}
\item $I^p_{CI}(Real) = I^p_{IA}(Real)$
\item $\forall p \in P^p$; $I^p_{CI}(p)= I^p_{IA}(p)$
\item $\forall f \in F^p \setminus \{\mathbf{1}\}$; $I^p_{CI}(f) = U^p_{CI} \times U^p_{CI} \mapsto U^p_{CI}$ such that $ I^p_{CI}(f)(i_1, i_2)= i_1 \; f_{CI} \; i_2$ where the definition of $f_{CI}$ is:
\begin{itemize}
\item[$\bullet$] $\langle l_1, h_1 \rangle \oplus_{CI} \langle l_2, h_2 \rangle = \langle l_1 + l_2, h_1 + h_2 \rangle $.
\item[$\bullet$] $\langle l_1, h_1 \rangle \ominus_{CI} \langle l_2, h_2 \rangle = \langle l_1 - h_2, h_1 - l_2 \rangle $.
\item[$\bullet$] Operation $i_1 \otimes_{CI} i_2$ is defined using case analysis on the types of $i_1$ and $i_2$. First, the intervals are classified into the following:
\begin{itemize}
\item $P = \{\langle a, b \rangle | a \ge 0 \wedge b > 0 \}$
\item $N = \{\langle a, b \rangle | b \le 0 \wedge a < 0 \}$
\item $M = \{\langle a, b \rangle | a < 0 < b \}$
\item $Z = \{\langle a, b \rangle\}$
The definition of $\otimes_{CI}$ is given in Table ~
\begin{center}
\begin{tabular}{ | c | c | c |}
\hline
Class of $\langle l_1, h_1 \rangle$ & Class of $\langle l_2, h_2 \rangle$ & $\langle l_1, h_1 \rangle \otimes_{CI} \langle l_2, h_2 \rangle$ \\ \hline
P & P & $\langle l_1 \times l_2, h_1 \times h_2 \rangle $ \\ \hline
P & M & $\langle h_1 \times l_2, h_1 \times h_2 \rangle $ \\ \hline
P & N & $\langle h_1 \times l_2, l_1 \times h_2 \rangle $ \\ \hline
M & P & $\langle l_1 \times h_2, h_1 \times h_2 \rangle $ \\ \hline
M & M & $\langle \min (l_1 \times h_2, h_1 \times l_2), max (l_1 \times l_2, h_1 \times h_2) \rangle $ \\ \hline
M & N & $\langle h_1 \times l_2, l_1 \times l_2 \rangle $ \\ \hline
N & P & $\langle l_1 \times h_2, h_1 \times l_2 \rangle $ \\ \hline
N & M & $\langle l_1 \times h_2, l_1 \times l_2 \rangle $ \\ \hline
N & N & $\langle h_1 \times h_2, l_1 \times l_2 \rangle $ \\ \hline
Z & P, N, M, Z & $\langle 0, 0 \rangle $ \\ \hline
P, N, M & Z & $\langle 0, 0 \rangle $ \\ \hline
\end{tabular}
\end{center}
\end{itemize}

\end{itemize}
\item $I^p_{CI}(\mathbf{1}) = \langle 1,1\rangle $
\item $\forall v \in V$; $I^p_{CI} \in U^p_{CI}$
\end{enumerate}
Theory $T^p_{CI} = \{M^p_{CI}| M^p_{CI} \text{ is a model over intervals}\}$. Each model differs to another by the mapping from variables to intervals. As a consequence, one assignment from variables to intervals can be used to describe an model. We denote $\Pi^p_{CI}$ as the model represented by $\Pi = \{x \in \langle l, h\rangle  | v \in V\}$. 
\begin{theorem}
CI is an IA.
\end{theorem}
\begin{proof}
Easy.
\end{proof}
\section{Affine Interval}
Affine Interval use the formula $a_0 + \sum\limits_{i=1}^{n} a_i\epsilon_i$ to represent the interval $\langle a_0 - \sum\limits_{i=1}^{n}|a_i|, a_0 + \sum\limits_{i=1}^{n}|a_i| \rangle$ with $a_i \in \mathbb{R}$ for $i = 0, 1, \cdots$. For example, the affine interval form of $(x \in) \langle 2, 4 \rangle$ and $(y \in) \langle 0, 2 \rangle$ is $3 + \epsilon_1$ and $1 + \epsilon_2$ respectively, thus:

\begin{center}
$\begin{aligned}[t]
    x^2 - x \times y  &= (3 + \epsilon_1)^2 - (3 + \epsilon_1) \times (1 + \epsilon_2)\\
     &= 9 + 6\epsilon_1 + \epsilon_1^2- (3 + 3\epsilon_2 + \epsilon_1 + \epsilon_1\epsilon_2) \\
     &= 6 + 5\epsilon_1 - 3\epsilon_2 + \epsilon_1^2 + \epsilon_1\epsilon_2
\end{aligned}$
\end{center}
Types of affine interval vary by choices of estimating multiplications $\epsilon_1^2$ and $\epsilon_1\epsilon_2$:
\begin{enumerate}
\item AA \cite{Comba93affinearithmetic, Stolfi97self-validatednumerical} replaces $\epsilon_1\epsilon_2$ by a fresh noise symbol. 
\item AF1 and AF2 \cite{Messine_extensionsof} prepares a fixed noise symbol for any $\epsilon_1\epsilon_2$.
\item EAI \cite{Ngoc:2009:ORE:1685167.1685421} replaces $\epsilon_1\epsilon_2$ by $\langle -1, 1 \rangle\epsilon_1$ or $\langle -1, 1 \rangle\epsilon_2$.
\item AF2 \cite{Messine_extensionsof} replaces $\epsilon_1^2$ by the fixed noise symbols $\epsilon_+$ or $\epsilon_-$.
\end{enumerate}
A model $M^p_{AF2} = (U^p_{AF2}, I^p_{AF2})$ over intervals contains a set of all intervals $U^p_{AF2} = \{a_0 + \sum\limits_{i=1}^{n}a_i\epsilon_i + a_{n+1}\epsilon_+ + a_{n+2}\epsilon_- + a_{n+3}\epsilon_{\pm} | \forall i \in \{0, 1, \cdots, n+3\}; a_i \in \mathbb{R}\}$ and a map $I^p_{AF2}$ that satisfies the following conditions.
\begin{enumerate}
\item $I^p_{AF2}(Real) = U^p_{AF2}$
\item $\forall p \in P^p$; $I^p_{AF2}(p)= U^p_{AF2} \times U^p_{AF2} \mapsto \{true, false\}$ such that $I^p_{AF2}(p)(a_0 + \sum\limits_{i=1}^{n}a_i\epsilon_i + a_{n+1}\epsilon_+ + a_{n+2}\epsilon_- + a_{n+3}\epsilon_{\pm}, b_0 + \sum\limits_{i=1}^{n}b_i\epsilon_i + b_{n+1}\epsilon_+ + b_{n+2}\epsilon_- + b_{n+3}\epsilon_{\pm}) = I^p_{AI}(p)(\langle a_0 - \sum\limits_{i=1}^{n}|a_i| - a_{n+2} - a_{n+3}, a_0 + \sum\limits_{i=1}^{n}|a_i| + a_{n+1} + a_{n+3} \rangle, \langle b_0 - \sum\limits_{i=1}^{n}|b_i| - b_{n+2} - b_{n+3}, b_0 + \sum\limits_{i=1}^{n}|b_i| + b_{n+1} + b_{n+3} \rangle)$
\item $\forall f \in F^p \setminus \{\mathbf{1}\}$; $I^p_{AF2}(f) = U^p_{AF2} \times U^p_{AF2} \mapsto U^p_{AF2}$ such that $ I^p_{AF2}(f)(i_1, i_2)= i_1 \; f_{AF2} \; i_2$ where the definition of $f_{AF2}$ is as following. Let $i_1 = a_0 + \sum\limits_{i=1}^{n}a_i\epsilon_i + a_{n+1}\epsilon_+ + a_{n+2}\epsilon_- + a_{n+3}\epsilon_{\pm}$ and $i_2 = b_0 + \sum\limits_{i=1}^{n}b_i\epsilon_i + b_{n+1}\epsilon_+ + b_{n+2}\epsilon_- + b_{n+3}\epsilon_{\pm}$, then:
\begin{itemize}
\item $i_1 \oplus_{AF2} i_2 = a_0 + b_0 + \sum\limits_{i=1}^{n}(a_i + b_i)\epsilon_i + (a_{n+1} + b_{n+1})\epsilon_+ + (a_{n+2} + b_{n+2})\epsilon_- + (a_{n+3} + b_{n+3})\epsilon_\pm$.
\item $i_1 \ominus_{AF2} i_2 = a_0 - b_0 + \sum\limits_{i=1}^{n}(a_i - b_i)\epsilon_i + (a_{n+1} + b_{n+1})\epsilon_+ + (a_{n+2} + b_{n+2})\epsilon_- + (a_{n+3} + b_{n+3})\epsilon_\pm$.
\item $i_1 \otimes_{AF2} i_2 = a_0b_0 + \sum\limits_{i=1}^{n}(a_0b_i + a_ib_0)\epsilon_i + K_1\epsilon_+ + K_2\epsilon_-  + K_3\epsilon_\pm$, where:

$K_1 = \sum\limits_{i=1,a_ib_i>0}^{n+3}a_ib_i + \left\{ 
  \begin{array}{l l}
    a_0b_{n+1} + a_{n+1}b_0 & \quad \text{if } a_0 \ge 0 \text{ and } b_0 \ge 0\\
    a_0b_{n+1} - a_{n+2}b_0 & \quad \text{if } a_0 \ge 0 \text{ and } b_0 < 0\\
    -a_0b_{n+2} + a_{n+1}b_0 & \quad \text{if } a_0 < 0 \text{ and } b_0 \ge 0\\
    -a_0b_{n+2} - a_{n+2}b_0 & \quad \text{if } a_0 < 0 \text{ and } b_0 < 0\\
  \end{array} \right.$
  
$K_2 = \sum\limits_{i=1,a_ib_i<0}^{n+3}a_ib_i + \left\{ 
  \begin{array}{l l}
    a_0b_{n+2} + a_{n+2}b_0 & \quad \text{if } a_0 \ge 0 \text{ and } b_0 \ge 0\\
    a_0b_{n+2} - a_{n+1}b_0 & \quad \text{if } a_0 \ge 0 \text{ and } b_0 < 0\\
    -a_0b_{n+1} + a_{n+2}b_0 & \quad \text{if } a_0 < 0 \text{ and } b_0 \ge 0\\
    -a_0b_{n+1} - a_{n+1}b_0 & \quad \text{if } a_0 < 0 \text{ and } b_0 < 0\\
  \end{array} \right.$  

$K_3 = \sum\limits_{i=1}^{n+3}\sum\limits_{j=1,j \neq i}^{n+3}|a_ib_j| + |a_0|b_{n+3} + a_{n+3}|b_0|$
\end{itemize}
\item $I^p_{AF2}(\mathbf{1}) = 1$
\item $\forall v \in V$; $I^p_{AF2} \in U^p_{AF2}$
\end{enumerate}
Theory $T^p_{AF2} = \{M^p_{AF2}| M^p_{AF2} \text{ is a model over intervals}\}$. Each model differs to another by the mapping from variables to intervals. As a consequence, one assignment from variables to intervals can be used to describe an model. We denote $\Pi^p_{CI}$ as the model represented by $\Pi = \{x \in \langle l, h\rangle  | v \in V\}$. 
\begin{theorem}
AF2 is an IA.
\end{theorem}
\begin{proof}
Easy.
\end{proof}

\begin{comment}
We design an SMT solver named raSAT~\footnote{\url{http://www.jaist.ac.jp/~mizuhito/tools/rasat.html}} to solve polynomial constraint. 

Although an $O.T$ refinement loop is enough to implement an ICP based SMT solver, 
we extend it as {\bf raSAT} (SAT by refinement of approximations) loop to accelerate SAT detection 
by adding $U.T$, which works as in Fig.~\ref{fig:OTrefine}~(b). 
\begin{enumerate}
\item When an over-approximation theory $O.T$ detects $O.T$-UNSAT (resp. $O.T$-valid), 
answer UNSAT (resp. SAT). 
\item When an under-approximation theory $U.T$ detects $U.T$-SAT, answer SAT. 
\item If neither holds, a refinement is applied. 
\end{enumerate}

Our design of an SMT solver {\bf raSAT} applies two heuristic features. 
\begin{itemize}
\item Incremental widening intervals, and incremental deeping search 
(Section~\ref{sec:incsearch}). 
\item 
Heurstic measures {\em SAT-likelyhood} and {\em sensitivity}, 
for selection of a variable to decompose and a box to explore. 
(Section~\ref{sec:SATheuristics}). 
\end{itemize} 

{\bf raSAT} also prepares various interval arithmetic as $O.T$ as in Section~\ref{sec:approximation}, 
whereas currently only random tesing (\emph{k-random ticks}, 
which consists of periodical $k$-test instances with a random offset) is prepared as $U.T$. 

A typical theory for $O.T$ and $U.T$ are an interval arithmetic and testing, respectively. 
We say {\em IA-valid}, {\em IA-SAT}, and {\em IA-UNSAT}, when it is $O.T$-valid, $O.T$-SAT, and 
$O.T$-UNSAT, respectively. 
Similarly, we say {\em test-SAT} and {\em test-UNSAT}, when it is $U.T$-SAT and $U.T$-UNSAT, respectively. 
Note that either IA-valid or test-SAT implies SAT, and IA-UNSAT implies UNSAT, 
whereas IA-SAT and test-UNSAT can conclude neither. 


%We instantiate testing to $U.T$ in Section~\ref{sec:raSATloop}. 
%%%%%%%%%%%%%%%%
\suppress{
\begin{definition}\label{def:testing}
%For $\exists x_1 \in (a_1,b_1) \cdots x_n \in (a_n,b_n). \bigwedge \limits_{i=1}^m f_i(x_1,\cdots,x_n) > 0$, 
Let $M = \bigwedge \limits_{i=1}^m x_i \in (a_i,b_i)$ and 
${\mathcal P} = \bigwedge \limits_{i=1}^m f_i(x_1,\cdots,x_n) > 0$. 
%
Let a choice function $\theta : (\Real \times \Real)^n \rightarrow \Real^n$ 
such that $\theta(M) \in (a_1,b_1) \times \cdots \times (a_n,b_n)$. 
Testing is a finite set $\Theta$ of choice functions. Then, we say 
\begin{itemize}
\item ${\mathcal P}$ is \emph{Test-SAT} under $M$ if $\theta(M)$ holds ${\mathcal P}$ 
for some $\theta \in \Theta$, and 
\item ${\mathcal P}$ is \emph{Test-UNSAT} under $M$ if $\theta(M)$ never holds ${\mathcal P}$ 
for each $\theta \in \Theta$. 
\end{itemize} 
%We denote $I \models_{test(\theta)} P$ if $\bigwedge \limits_{i=1}^m f_i(\theta(I)) > 0$ holds.
\end{definition}
}
%%%%%%%%%%%%%%%%


{\bf raSAT} prepares various Affine intervals, adding to classical interval (CI)~\cite{moore}, 
which keep lower and upper bounds. The weakness of CI is loss of dependency 
among values. For instance, $x - x$ is evaluated to $(-2,2)$ for $x \in (2,4)$. 

Affine Interval~\cite{af,comba93} introduces \emph{noise symbols} $\epsilon$, 
which are interpreted as values in $(-1,1)$. 
For instance, $x = 3 + \epsilon$ describes $x \in (2,4)$, and 
$x - x = (3 + \epsilon) - (3 + \epsilon)$ is evaluated to $0$. 
The drawback is that the multiplication without dependency might be less precise than CI.
Affine intervals also cannot represent infinite intervals, e.g., $(0,\infty)$, 
since it becomes $\infty + \infty~\epsilon$. 
Forms of affine intervals vary by choices how to approximate multiplications. They are,
\begin{enumerate}[(i)]
\item $\epsilon \epsilon'$ is replaced with a fresh noise symbol 
($AF$)~\cite{StolfiThesis,comba93}, 
\item $\epsilon \epsilon'$ is reduced to the fixed error noise symbol 
$\epsilon_{\pm}$ ($AF_1$ and $AF_2$) \cite{af},
\item $\epsilon \epsilon'$ is replaced with $(-1,1) \epsilon$ 
(or $(-1,1) \epsilon'$) ($EAI$)~\cite{ngocsefm},
\item $\epsilon \epsilon$ is reduced to fixed noise symbols 
$\epsilon_+$ or $\epsilon_{-}$ ($AF_2$) \cite{af}, 
\item Chebyshev approximation of $x^2$ introduces a noise symbol $|\epsilon|$ 
as an absolute value of $\epsilon$ with 
$\epsilon \epsilon = |\epsilon| |\epsilon| = |\epsilon| + (-\frac{1}{4}, 0)$ and
$\epsilon |\epsilon| = \epsilon + (-\frac{1}{4}, \frac{1}{4})$ \cite{tapas12}. 
%(Fig.~\ref{fig:chevabs}). 
%\item keeping products of noise symbols up to degree $2$ ($\epsilon_i \epsilon_j$),
\end{enumerate} 

%%%%%%%%%%%%%%%%%
\suppress{
\begin{remark}
For Affine intervals, \emph{sensitivity}~\cite{ngocsefm} of a variable
is a possible range of the absolute value of the coefficient of its corresponding $\epsilon$. 

%In Example~\ref{examp:sensitivity}, $CAI$ estimates the coefficient of $|\epsilon_1|$ as $\textbf{3}$, 
%which has the largest sensitivity and indicates $x$ the most influencial. 

Note that Affine interval works only for bounded intervals. 
For instance, $\infty + \infty \epsilon$ represents $(-\infty,\infty)$, which says nothing. 
Narrowing intervals as an incremental search (Section~\ref{sec:incsearch}) partilly depends on this fact. 
That is, if $\pm \infty$ is contained in an interval, first give finite upper/lower bounds and search 
within these bounds using an Affine interval. If UNSAT is concluded, then enlarge to the whole intervals 
using CI. 
\end{remark}
}
%%%%%%%%%%%%%%%%%


\begin{example} \label{examp:sensitivity}
Let $f = x^3 - 2xy$ with $x = (0,2)$ ($x = 1 + \epsilon_1$) and $y=(1,3)$ ($y = 2+\epsilon_2$), 
we have,
\begin{itemize}
\item $AF_2$ estimates the range of $f$ as 
$-3 - \epsilon_1 - 2\epsilon_2 + 3\epsilon_+ + 3\epsilon_{\pm}$, thus $(-9,6)$,
\item $CAI$ estimates the range of $f$ as 
$(-4,-\frac{11}{4}) + (-\frac{1}{4}, 0)\epsilon_1 - 2\epsilon_2 + \textbf{3}|\epsilon_1| + (-2,2)\epsilon_{\pm}$, 
thus $(-8,4.5)$.
\end{itemize}
\end{example}



%%%%%%%%%%%%%%%%%%%%%%%%%%
\suppress{
\begin{figure}[ht]
\begin{minipage}[b]{1.0\linewidth}
\centering
\begin{tabular}{ll}
\includegraphics[height=1.6in,width=1.7in]{chev1.pdf} &
\includegraphics[height=1.6in,width=1.7in]{chev2.pdf}
\end{tabular}
\caption{Chebyshev approximation of $x^2$ and $x~|x|$}
\label{fig:chevabs}
\end{minipage}
\end{figure}

$CAI$ \cite{tapas12} consists of (ii) and (v), which keeps better precision than iv)
for multiplicatins of the same variables, e.g., Taylor expansion. 
%To improve precision in estimating upper and lower bounds of polynomials, we apply 
%\textbf{Affine Arithmetic} such as $AF_1$, $AF_2$ \cite{af}, $CAI$ ~\cite{tapas12} 
%instead of Classical Interval \cite{moore}. 
%Note that upper and lower bounds estimated by IA are over-approximation bounds of polynomials.

}
%%%%%%%%%%%%%%%%%%%%%%%%%%
\suppress{
\begin{definition}
%For $\exists x_1 \in (a_1,b_1) \cdots x_n \in (a_n,b_n). \bigwedge \limits_{i=1}^m f_i(x_1,\cdots,x_n) > 0$, 
Let $M = \bigwedge \limits_{i=1}^m x_i \in (a_i,b_i)$ and 
${\mathcal P} = \bigwedge \limits_{i=1}^m f_i(x_1,\cdots,x_n) > 0$. 
%
Let $\delta_i^l$ and $\delta_i^u$ be lower and upper bounds of $f_i(x_1,\cdots,x_n)$ 
estimated by IA for $x_i \in (a_i,b_i)$. Then, we say 
%
%\vspace*{0.5em}
\begin{itemize}
\item ${\mathcal P}$ is \emph{IA-VALID} under $M$, if IA evaluates 
$~\forall i \in [1,m].~\delta_i^l > 0$,
%\vspace*{0.33em}
\item ${\mathcal P}$ is \emph{IA-UNSAT} under $M$, 
$~\exists i \in [1,m].~\delta_i^u \leq 0$, and 
\item ${\mathcal P}$ is \emph{IA-SAT} under $M$, if 
$(\exists j \in [1,m].~\delta_j^l \leq 0)\; \wedge \; 
	(\bigwedge \limits_{i=1}^m \delta_i^u > 0)$.
\end{itemize} 
\end{definition}

IA-VALID and IA-UNSAT safely reason satisfiability (SAT) and unsatisfiability (UNSAT), 
respectively. However, IA-SAT cannot conclude SAT. 
}


\suppress{
I. Selecting API for testing:
  (1) Difficulty first by SAT-likelihood.   
  (2) Easy first by SAT-likelihood
  (10) Random.,
II. Selecting Variable:
  (8) With sensitivity
  (9) Without sensitivity - Random: 
III. Selecting box:
  (3) SAT-directed using IA-Testing.
  (4) UNSAT-directed using IA-Testing.
  (5) SAT-directed using SAT-likelihood
  (6) UNSAT-directed using SAT-likelihood
  (7) Random
}
\end{comment} 
