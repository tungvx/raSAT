\chapter{Experiments}
This chapter is going to present the experiments results which reflect how effective our designed strategies are. In addition, comparison between raSAT, Z3 and iSAT3 will be also shown. The experiments were done on a system with  Intel Xeon E5-2680v2 2.80GHz and 4 GB of RAM. In the experiments, we exclude the problems which contain equalities because currently raSAT focuses on inequalities only.

\section{Experiments on Strategy Combinations} \label{sec:expstrategy}

We perform experiments only on Zankl, and Meti-Tarski families. 


Our combinations of strategies mentioned in Section~\ref{sec:SATheuristics} are, 

\medskip
{\centering
\begin{tabular}{l|l|l}
Selecting a test-UNSAT API~~ & Selecting a box (to explore): & 
Selcting a variable: \\  % (for testing and decomposition)
\hline

(1) Least SAT-likelyhood. & 
(3) Largest number of SAT inequalities.~~ & 
(8) Largest sensitibity. \\

(2) Largest SAT-likelyhood. & 
(4) Least number of SAT inequalities. & \\

& (5) Largest SAT-likelyhood. & \\

& (6) Least SAT-likelyhood. & \\

(10) Random. & (7) Random. & (9) Random. \\
\end{tabular}
}
\medskip

Table~\ref{tab:rasat-experiments} shows the experimental results of above mentioned combination. 
The timeout is set to 500s, and each time is the total of successful cases 
(either SAT or UNSAT). 

Note that (10)-(7)-(9) means all random selection. 
Generally, the combination of (5) and (8) show the best results, 
though the choice of (1),(2), and (10) shows different behavior on benchmarks. 
We tentatively prefer (1) or (10), but it needs to be investigated further. 

\begin{table*}[t]
\centering
\resizebox{\columnwidth}{!}{%
\begin{tabular}{ | l | r | r  r | r | r  | r | r | r | r | r | r |r | r |}
\hline
    \multicolumn{1}{|l|}{Benchmark} & 
    \multicolumn{3}{c|}{(1)-(5)-(8)} & \multicolumn{2}{c|}{(1)-(5)-(9)} & 
    \multicolumn{2}{c|}{(1)-(6)-(8)} & \multicolumn{2}{c|}{(1)-(6)-(9)} &
    \multicolumn{2}{c|}{(10)-(5)-(8)} & \multicolumn{2}{c|}{(10)-(6)-(8)} 
\\
\hline
 Matrix-1 (SAT) & 20 & 132.72 & (s) & 21 & 21.48 & 19 & 526.76 & 18 & 562.19 & 21 & 462.57 & 19 & 155.77 
\\
\hline
 Matrix-1 (UNSAT) & 2 & 0.00 && {\bf 3} & 0.00 & {\bf 3} & 0.00 & {\bf 3} & 0 & {\bf 3} & 0.00 
& {\bf 3} & 0.00 
\\
\hline
 Matrix-2,3,4,5 (SAT) & {\bf 11} & 632.37 && 1 & 4.83 & 0 & 0.00 & 1 & 22.50 & 9 & 943.08 & 1 & 30.48 
\\
\hline
 Matrix-2,3,4,5 (UNSAT) & 8 & 0.37 && 8 & 0.39 & 8 & 0.37 & 8 & 0.38 & 8 & 0.38 & 8 & 0.38 
\\
\hline
\hline
    \multicolumn{1}{|l|}{Benchmark} & 
    \multicolumn{3}{c|}{(2)-(5)-(8)} & \multicolumn{2}{c|}{(2)-(5)-(9)} & 
    \multicolumn{2}{c|}{(2)-(6)-(8)} & \multicolumn{2}{c|}{(2)-(6)-(9)} & 
    \multicolumn{2}{c|}{(2)-(7)-(8)} & \multicolumn{2}{c|}{(10)-(7)-(9)} \\
\hline
 Matrix-1 (SAT) & {\bf 22} & 163.47 & (s) & 19 & 736.17 & 20 & 324.97 & 18 & 
1068.40 & 21 & 799.79 & 21 & 933.39 
\\
\hline
 Matrix-1 (UNSAT) &     2 & 0 && 2 & 0.00 & 2 & 0.00 & 2 & 0.00 & 2 & 0.00 & 2 & 0.00 
\\
\hline
 Matrix-2,3,4,5 (SAT) &     5 & 202.37 && 1 & 350.84 & 1 & 138.86 & 0 & 0.00 & 0 & 0.00 & 0 & 0.00 
\\
\hline
 Matrix-2,3,4,5 (UNSAT) &     8 & 0.43 && 8 & 0.37 & 8 & 0.40 & 8 & 0.38 & 8 & 0.37 & 8 & 0.38 
\\
\hline
\hline
    \multicolumn{1}{|l|}{Benchmark} & 
    \multicolumn{3}{c|}{(1)-(3)-(8)} & \multicolumn{2}{c|}{(1)-(4)-(8)} & 
    \multicolumn{2}{c|}{(2)-(3)-(8)} & \multicolumn{2}{c|}{(2)-(4)-(8)} & 
    \multicolumn{2}{c|}{(10)-(3)-(8)} & \multicolumn{2}{c|}{(10)-(4)-(8)} \\
\hline
 Matrix-1 (SAT) & 20 & 738.26 & (s) & 21 & 1537.9 & 18 & 479.60 & 21 & 867.99 & 20 & 588.78 & 19 & 196.21 
\\
\hline
 Matrix-1 (UNSAT) & 2 & 0.00 && 2 & 0.00 & 2 & 0.00 & 2 & 0.00 & 2 & 0.00 & 2 & 0.00 
\\
\hline
 Matrix-2,3,4,5 (SAT) & 0 & 0.00 && 2 & 289.17 & 1 & 467.12 & 1 & 328.03 & 1 & 195.18 & 2 & 354.94 
\\
\hline
 Matrix-2,3,4,5 (UNSAT) & 8 & 0.36 && 8 & 0.36 & 8 & 0.34 & 8 & 0.37 & 8 & 0.37 & 8 & 0.39 
\\
\hline
\end{tabular}
}
\bigskip
\resizebox{\columnwidth}{!}{%
\begin{tabular}{ | l | r | r  r | r  | r | r | r | r | r |}
\hline
    \multicolumn{1}{|l|}{Benchmark} & 
    \multicolumn{3}{c|}{(1)-(5)-(8)} & \multicolumn{2}{c|}{(1)-(5)-(9)} & 
    \multicolumn{2}{c|}{(10)-(5)-(8)} & \multicolumn{2}{c|}{(10)-(7)-(9)} \\
\hline
    Meti-Tarski (SAT, 3528) & 3322 & 369.60 & (s) & 3303 & 425.37 & {\bf 3325} & 653.87 & 3322 & 642.04 
\\
\hline
    Meti-Tarski (UNSAT, 1573) & 1052 & 383.40 && 1064 & 1141.67 & {\bf 1100} & 842.73 & 1076 & 829.43 
\\
\hline
\end{tabular}
}
\medskip
\caption{Combinations of {\bf raSAT} strategies on NRA/Zankl,Meti-Tarski benchmark} 
\label{tab:rasat-experiments}
\end{table*}


\section{Comparison with other SMT Solvers}

We compare {\bf raSAT} with other SMT solvers on NRA benchmarks, Zankl and Meti-Tarski. 
The timeouts for Zankl and Meti-tarski are $500s$ and $60s$, respectively. 
For {\bf iSAT3}, ranges of all variables are uniformly set to be in the range $[-1000, 1000]$
(otherwise, it often causes segmentation fault). 
Thus, UNSAT detection of {\bf iSAT3} means UNSAT in the range $[-1000, 1000]$, 
while that of {\bf raSAT} and {\bf Z3 4.3} means  UNSAT over $[-\infty, \infty]$. 

Among these SMT solvers, {\bf Z3 4.3} shows the best performance. 
However, if we closely observe, there are certain tendency. 
{\bf Z3 4.3} is very quick for small constraints, i.e., with 
short APIs (up to $5$) and a small number of variables (up to $10$). 
{\bf raSAT} shows comparable performance on SAT detection with 
longer APIs (larger than $5$) and a larger number of variables (more than $10$), 
and sometimes outforms for SAT detection on vary long constraints 
(APIs longer than $40$ and/or more than $20$ variables). 
Such examples appear in Zankl/matrix-3-all-*, matrix-4-all-*, and matrix-5-all-* 
(total 74 problems), and {\bf raSAT} solely solves 
\begin{itemize}
\item[$\bullet$] {\em matrix-3-all-2} (47 variables, 87 APIs, and max length of an API is 27), 
\item[$\bullet$] {\em matrix-3-all-5} (81 variables, 142 APIs, and max length of an API is 20), 
\item[$\bullet$] {\em matrix-4-all-3} (139 variables, 244 APIs, and max length of an API is 73), and 
\item[$\bullet$] {\em matrix-5-all-01} (132 variables, 276 APIs, and max length of an API is 47). 
\end{itemize}
Note that, for Zankl, when UNSAT is detected, it is detected very quickly. 
This is because SMT solvers detects UNSAT only when they find small UNSAT cores, 
without tracing all APIs. However, for SAT detection with ;arge problems, 
SMT solvers need to trace all problems. Thus, it takes much longer time. 

\begin{table*}[t]
\centering
\resizebox{\columnwidth}{!}{%
\begin{tabular}{ | l | r | r  r | r | r  | r | r | r | r | r | r |r | r |}
\hline
    \multicolumn{1}{|l|}{Benchmark} & 
    \multicolumn{5}{c|}{\bf raSAT} & \multicolumn{4}{c|}{\bf Z3 4.3} & \multicolumn{4}{c|}{\bf iSAT3} \\
\hline
    & \multicolumn{3}{|c|}{SAT} & \multicolumn{2}{|c|}{UNSAT} & \multicolumn{2}{|c|}{SAT} 
    & \multicolumn{2}{|c|}{UNSAT} & \multicolumn{2}{|c|}{SAT} & \multicolumn{2}{|c|}{UNSAT} \\
\hline
Zankl/matrix-1 (53) & 20 & 132.72 & (s) & 2 & 0.00 & 41 & 2.17 & 12 & 0.00 & 11 & 4.68 & 3 & 0.00 \\
\hline
Zankl/matrix-2,3,4,5 (98) & 11 & 632.37 && 8 & 0.37 & 13 & 1031.68 & 11 & 0.57 & 3 & 196.40 & 12 & 8.06 \\
\hline 
Meti-Tarski (3528/1573) & 3322 & 369.60 && 1052 & 383.40 & 3528 & 51.22 & 1568 & 78.56 & 2916 & 811.53 & 
1225 & 73.83 \\
\hline
\end{tabular}
}
\medskip 
\caption{Comparison among SMT solvers} \label{tab:comparison}
\end{table*}


\section{Experiments with QE-CAD Benchmark}
We also did experiments on QE-CAD problems provided by Mohab Safey El Din\footnote{\url{http://www-polsys.lip6.fr/~safey/}} - LIP6 who is working on QE-CAD simplification (checking SAT/UNSAT only and the complexity is from DEXP to EXP) and the QEPCAD problems collected by Hidenao Iwane\footnote{\url{https://github.com/hiwane/qepcad/}}. 

The benchmarks from LIP6 are all unsatisfiable and contain long polynomials (in terms of monomials number). From the experiments with raSAT and Z3, these benchmarks show that they are very difficult to be solved (Table~\ref{exp:lip6}). These benchmarks may be suitable targets for our UNSAT core strategies in the future. 

\begin{table} \label{exp:hiwane}
\begin{center}
\begin{tabular}{|c|c|c|c|c|c|c|c|}
\hline
\multicolumn{4}{|c|}{raSAT} & \multicolumn{4}{c|}{Z3}\\ \hline
\multicolumn{2}{|c|}{SAT}  & \multicolumn{2}{c|}{UNSAT} & \multicolumn{2}{c|}{SAT} & \multicolumn{2}{c|}{UNSAT} \\ \hline
156 & 244.6(s) & 2 & 0.03 (s) & 205 & 1.12 (s) & 22 & 0.05 (s) \\ \hline
\end{tabular}
\end{center}
\caption{Experiments on 217 QEPCAD problems from Hidenao Iwane}
\end{table}

In terms of QEPCAD benchmarks, while Z3 solves all $217$ problems, raSAT solves $156/205$ satisfiable and $2/22$ unsatisfiable benchmarks. Apart from a number of problems which raSAT has errors in parsing, the unsolved ones are mostly unsatisfiable. This can be tackled by strategies on UNSAT core which is left for our future works. 
\begin{table} \label{exp:lip6}
\begin{center}
\begin{tabular}{| c | c | c | c | c |}
\hline
\multirow{2}{*}{Problem} & \multicolumn{2}{ c }{raSAT} & \multicolumn{2}{ |c| }{Z3}\\ \cline{2-5}& Time (s) & Result & Time (s) & Result\\ \hline
f23 & 3600 & Timeout & 3599.55 & Timeout\\ \hline
f22 & 3600 & Timeout & 3599.51 & Timeout\\ \hline
pol & 3600 & Timeout & 0.02499 & UNSAT\\ \hline
f13 & 3600 & Timeout & 3599.17 & Timeout\\ \hline
pol1 & 3600 & Timeout & 3601.61 & Timeout\\ \hline
f12 & 3600 & Timeout & 3599.27 & Timeout\\ \hline
\end{tabular}
\caption{Experiments on problems from LIP6}
\end{center}
\end{table}


%%%%%%%%%%%%%%%%%