\chapter{Preliminaries}
\section{Abstract DPLL}
In propositional logic, we have a set of \emph{propositional symbols} $P$ and every $p \in P$ is called an \emph{atom}. A \emph{literal} $l$ is either $p$ or $\neg p$ with $p \in P$. The \emph{negation} $\neg l$ of a literal $l$ is $\neg p$ if $l$ is $p$, or $p$ if $l$ is $\neg p$. A disjunction $l_1 \vee \cdots \vee l_n$ of literals is said to be a \emph{clause}. A \emph{Conjuctive Normal Form (CNF) formula} is a conjunction of clauses $C_1 \wedge \cdots \wedge C_n$. If $C = l_1 \vee \cdots \vee l_n$ is a clause, $\neg C$ is used to denote the CNF formula $\neg l_1 \wedge \cdots \wedge \neg l_n$

An (partial) \emph{assignment} $M$ is a set of literals such that $l \in M$ and $\neg l \in M$ for no literal $l$. A literal $l$ is undefined in $M$ if neither $l \in M$ nor $\neg l \in M$. If $l \in M$, $l$ is said to be true in M. On the other hand if $\neg l \in M$, we say that $l$ is false in $M$. A clause is true in $M$ if at least one of its literals is in $M$. An assignment $M$ satisfies a CNF formula $F$ (or $F$ is satisfied by $M$) if all clauses of $F$ is true in $M$ which is denoted as $M \models F$. Given two CNF formula $F$ and $F'$, we write $F \models F'$ if for any assignment $M$, $M \models F$ implies $M \models F'$. The formula $F$ is unsatisfiable if there is no assignment $M$ such that $M \models F$.

Abstract Davis–Putnam–Logemann–Loveland (DPLL) Procedure \cite{Nieuwenhuis05abstractdpll} searches for an assignment that satisfies a CNF formula. Each state of the procedure is either \emph{FailState} or a pair $M \parallel  F$ of an assignment $M$ and a CNF formula $F$. For the purpose of the procedure, $M$ is represented as a sequence of literals where each literal is optionally attached an \emph{annotation}, e.g. $l^d$ which basically means that the literal $l$ is selected to be included in the assignment by making a decision ($l$ is called a decision literal). An empty sequence is denoted by $\emptyset$. Each DPLL procedure is modeled by a collection of states and a binary relation $\Longrightarrow$ between states. Basic DPLL procedure is a transition system which contains the following four rules.
\begin{enumerate}
\item $\mathbf{UnitPropagate}$

$M \parallel F \wedge (C \vee l) \Longrightarrow Ml \parallel F \wedge (C \vee l)$ if $\left\{ 
  \begin{array}{l}
    M \models \neg C \\
    l\text{ is undefined in }M.
  \end{array} \right.$
  
\item $\mathbf{Decide}$

$M \parallel F \Longrightarrow Ml^d \parallel F$ if $\left\{ 
  \begin{array}{l}
    l \text{ or } \neg l \text{ occur in a clause of F} \\
    l\text{ is undefined in }M.
  \end{array} \right.$

\item $\mathbf{Fail}$

$M \parallel F \wedge C \Longrightarrow FailState$ if $\left\{ 
  \begin{array}{l}
    M \models \neg C \\
    l^d \in M \text{ for no literal }l.
  \end{array} \right.$  
  
\item $\mathbf{Backjump}$

$Ml^dM' \parallel F \wedge C \Longrightarrow Ml' \parallel F \wedge C$ if $\left\{ 
  \begin{array}{l}
    Ml^dM' \models \neg C,\text{ and there is some clause } C' \vee l'\\ 
    \text{such that }
    F \wedge C \models C' \vee l'
    \text{ and }M \models \neg C', l' \text{ is }\\
    \text{undefined in } M, \text{ and } l' \text{ or } \neg l' \text{ occurs in } F \text{ or in } \\Ml^dM'.
  \end{array} \right.$  
\end{enumerate}

Let $F$ be a given CNF formula. Starting with the state $\emptyset \parallel F$, basic DPLL procedure terminates with either $FailState$ (which is denoted as $\emptyset \parallel F \Longrightarrow^! FailState$) when $F$ is unsatisfiable, or a state $M \parallel F$ (which is denoted as $\emptyset \parallel F \Longrightarrow^! M \parallel F$) where $M$ satisfies $F$. Intuitively, the above four rules can be explained as following.

\begin{itemize}
\item $\mathbf{UnitPropagate}$: In order to satisfy $F \wedge (C \vee l)$, $C \vee l$ needs to be satisfied. Because all the literals in $C$ is false in current assignment $M$ ($M \models \neg C$), $l$ must be made true when extending $M$.
\item $\mathbf{Decide}$: This rule is applied when no more $\mathbf{UnitPropagation}$ can be applied. The annotation $d$ in $l^d$ denotes that if $Ml$ cannot be extended to satisfy $f$, $M\neg l$ needs to be explored further.
\item $\mathbf{Fail}$: When a clause is false in $M$ (conflict) and $M$ has no literal which is decided by making a decision (there is no more options to explored), the formula $F$ is unsatisfiable.
\item $\mathbf{Backjump}$: As same as in $\mathbf{Fail}$ rule, a conflict is detected. However, in $\mathbf{Backjump}$, because there exists some decision literal in the assignment, new possible assignments can be explored. The clause $C' \vee l'$ is called the backjump clause.
\end{itemize}

\begin{example}
For the formula $(\neg l_1 \vee l_2) \wedge (\neg l_3 \vee l_4) \wedge (\neg l_5 \vee \neg l_6) \wedge (l_6 \vee \neg l_5 \vee \neg l_2)$, the basic DPLL procedure proceeds as following:

\begin{align*}
\emptyset &\parallel (\neg l_1 \vee l_2) \wedge (\neg l_3 \vee l_4) \wedge (\neg l_5 \vee \neg l_6) \wedge (l_6 \vee \neg l_5 \vee \neg l_2) \Longrightarrow (\mathbf{Decide}) \\
\l_1^d &\parallel (\neg l_1 \vee l_2) \wedge (\neg l_3 \vee l_4) \wedge (\neg l_5 \vee \neg l_6) \wedge (l_6 \vee \neg l_5 \vee \neg l_2) \Longrightarrow (\mathbf{UnitPropagate}) \\
\l_1^dl_2 &\parallel (\neg l_1 \vee l_2) \wedge (\neg l_3 \vee l_4) \wedge (\neg l_5 \vee \neg l_6) \wedge (l_6 \vee \neg l_5 \vee \neg l_2) \Longrightarrow (\mathbf{Decide}) \\
\l_1^dl_2l_3^d &\parallel (\neg l_1 \vee l_2) \wedge (\neg l_3 \vee l_4) \wedge (\neg l_5 \vee \neg l_6) \wedge (l_6 \vee \neg l_5 \vee \neg l_2) \Longrightarrow (\mathbf{UnitPropagate}) \\
\l_1^dl_2l_3^dl_4 &\parallel (\neg l_1 \vee l_2) \wedge (\neg l_3 \vee l_4) \wedge (\neg l_5 \vee \neg l_6) \wedge (l_6 \vee \neg l_5 \vee \neg l_2) \Longrightarrow (\mathbf{Decide})\\
\l_1^dl_2l_3^dl_4l_5^d &\parallel (\neg l_1 \vee l_2) \wedge (\neg l_3 \vee l_4) \wedge (\neg l_5 \vee \neg l_6) \wedge (l_6 \vee \neg l_5 \vee \neg l_2) \Longrightarrow (\mathbf{UnitPropagate})\\
\l_1^dl_2l_3^dl_4l_5^dl_6 &\parallel (\neg l_1 \vee l_2) \wedge (\neg l_3 \vee l_4) \wedge (\neg l_5 \vee \neg l_6) \wedge (l_6 \vee \neg l_5 \vee \neg l_2) \Longrightarrow (\mathbf{Backjump})\\
\l_1^dl_2l_3^dl_4\neg l_5 &\parallel (\neg l_1 \vee l_2) \wedge (\neg l_3 \vee l_4) \wedge (\neg l_5 \vee \neg l_6) \wedge (l_6 \vee \neg l_5 \vee \neg l_2)
\end{align*}
\sloppy
In the $\mathbf{Bacjump}$ step of this example: $M$ is $l_1^dl_2l_3^dl_4$, $l^d$ is $l_5^d$, 
$M'$ is $l_6$,
$F$ is ${(\neg l_1 \vee l_2) \wedge (\neg l_3 \vee l_4) \wedge (\neg l_5 \vee \neg l_6)}$,
$C$ is $l_6 \vee \neg l_5 \vee \neg l_2$,
$C'$ is $\neg l_1$, and $l'$ is $\neg l_5$.
\end{example}

Additionally DPLL implementation can add the backjump clauses into the CNF formula as learnt clause (or lemmas), which is usually referred as \emph{conflict-driven learning}. Lemmas aim at preventing similar conflicts to occur in the future. When the conflicts are not likely to happen, the lemmas can be removed. The following two rules are prepared for DPLL.

\begin{enumerate}
\setcounter{enumi}{5}
\item $\mathbf{Learn}$

$M \parallel F \Longrightarrow M \parallel F \wedge C$ if $\left\{ 
  \begin{array}{l}
    \text{each atom in C appears in F} \\
    F \models C.
  \end{array} \right.$ 
  
\item $\mathbf{Forget}$

$M \parallel F \wedge C \Longrightarrow M \parallel F$ if $\left\{ 
  \begin{array}{l}
    \text{each atom in C appears in F} \\
    F \models C.
  \end{array} \right.$   
\end{enumerate}

\section{Satisfiability Modulo Theories - SMT}
\subsection{Syntax} \label{subsection:smt-syntax}
\begin{definition}
A \emph{signature} $\Sigma$ is a 4-tuple $(S, P, F, V, \alpha)$ consisting of a set $S$ of \emph{sorts}, a set $P$ of \emph{predicate symbols}, a set $F$ of \emph{function symbols}, a set $V$ of \emph{variables}, and a \emph{sorts map} $\alpha$ which associates symbols to their sorts such that
\begin{itemize}
\item for all $p \in P, \, \alpha(p)$ is a $n$-tuple argument sorts of $p$,
\item for all $f \in F, \, \alpha(f)$ is a $n$-tuple of argument and returned sorts of $f$, and
\item for all $v \in V, \, \alpha(v)$ represents the sort of variable $v$.
\end{itemize}
\end{definition}
A $\Sigma$-\emph{term} $t$ over the signature $\Sigma$ is defined as
\begin{center}
\begin{tabular}{l r l l}
$t$ &::=& $v$ & where $v \in V$\\ 
&$\mid$ & $f(t_1, \cdots, t_n)$ &where $f \in F$ with arity $n$
\end{tabular}
\end{center}



A $\Sigma$-\emph{formula} $\varphi$ over the signature $\Sigma$ is defined recursively as (we only focus on equantifier-free formulas):

\begin{center}
\begin{tabular}{l r l l}
$\varphi$ &::=& $p(t_1, \cdots, t_n)$ &where $p \in P$ with arity $n$ \\ 
&$\mid$ & $\bot \mid \neg \varphi_1$ &\\
&$\mid$ & $\varphi_1 \wedge \varphi_2 \mid \varphi_1 \vee \varphi_2$ & \\
&$\mid$ & $\varphi_1 \rightarrow \varphi_2 \mid \varphi_1 \leftrightarrow \varphi_2$ & \\
\end{tabular}
\end{center}

\subsection{Semantics}

\begin{definition}
Let $\Sigma = (S, P, F, V, \alpha)$ is a signature. A $\Sigma$-\emph{model} $M$ of $\Sigma$ is a pair $(U, I)$ in which $U$ is the \emph{universe} and $I$ is the \emph{interpretation} of symbols such that
\begin{itemize}
\sloppy
\item for all $s \in S$, $I(s) \subseteq U$ specifies the possible values of sort $s$,
\item for all $f \in F$, $I(f)$ is a function from $I(s_1) \times \cdots \times I(s_{n-1})$ to $I(s_n)$ with ${\alpha(f) = (s_1,\cdots, s_n)}$,
\item for all $p \in P$, $I(p)$ is a function from $I(s_1) \times \cdots \times I(s_n)$ to $\{0, 1\}$ where ${\alpha(p) = (s_1,\cdots, s_n)}$, and
\item for all $v \in V$, $I(v) \in I(\alpha(v))$
\end{itemize}
A $\Sigma$-\emph{theory} T is a (infinite) set of $\Sigma$-models. A theory $T'$ is a subset of theory $T$ if and only if $T' \subseteq T$.
\end{definition}

The interpretation of each predicate or function symbol is allowed to be not total, i.e. $I(p)$ or $I(f)$ are not necessarily total. We extend the universe of each model to include the symbol $\mathring{u}$ (unknown) which is prepared to indicate the result of undefined operations. For further convenience, we also define the following relations: $\mathring{u} < 0, 1$ and $\mathring{u} > 0, 1$ and the following arithmetic $1 - \mathring{u} = \mathring{u}$ which are useful when we evaluate the values of formulas containing logical connectives ($\wedge$, $\vee$, $\rightarrow$, $\leftrightarrow$, or $\neg$). 

\begin{definition} \label{def:smt-valuation}
Let $\Sigma = (S, P, F, V, \alpha)$ and $M=(U, I)$ are a signature and a $\Sigma-$model respectively. The \emph{valuation} of a $\Sigma$-term $t$ against $M$ which is denoted by $t^M$ is defined recursively as:
\begin{center}
\begin{tabular}{l c l l}
$v^M$ &=& $I(v)$ & where $v \in V$, and\\
$f^M(t_1, \cdots, t_n)$ &=& $\left\{ 
  \begin{array}{c l}
    I(f)(t_1^M, \cdots, t_n^M) & \quad \text{if } (t_1^M, \cdots, t_n^M) \in Dom(I(f))\\
    \mathring{u} & \quad \text{otherwise }\\
  \end{array} \right.$ & where $f \in F$. \\
\end{tabular}
\end{center}
Similarly, the valuation $\varphi^M$ of $\varphi$ is defined as:
\begin{center}
\begin{tabular}{l c l l}
$p^M(t_1,\cdots,t_n)$ &=& $\left\{ 
  \begin{array}{c l}
    I(p)(t_1^M,\cdots,t_n^M) & \quad \text{if } (t_1^M, \cdots, t_n^M) \in Dom(I(p))\\
    \mathring{u} & \quad \text{otherwise }\\
  \end{array} \right.$ & where $p \in P$, \\
$\bot^M$ &=& $0$ ,& \\
$(\neg \varphi_1)^M$ &=& $1 - \varphi_1^M$ ,& \\
$(\varphi_1 \wedge \varphi_2)^M$ &=& $\min(\varphi_1^M, \varphi_2^M)$, & \\
$(\varphi_1 \vee \varphi_2)^M$ &=& $\max(\varphi_1^M, \varphi_2^M)$, & \\
$(\varphi_1 \rightarrow \varphi_2)^M$ &=& $\max(1-\varphi_1^M, \varphi_2^M)$, and & \\
$(\varphi_1 \rightarrow \varphi_2)^M$ &=& $\left\{ 
  \begin{array}{c l}
    1 & \quad \text{if } \varphi_1^M = \varphi_2^M \\
    0 & \quad \text{otherwise }\\
  \end{array} \right.$. &
\end{tabular}
\end{center}
We say that $M$ satisfies $\varphi$ which is denoted by $\models_M \varphi$ iff $\varphi^M = 1$. If $\varphi^M = 0$, $\not\models_M \varphi$ is used to denote that $M$ does not satisfy $\varphi$.
\end{definition}

Given a signature $\Sigma$, a $\Sigma$-theory $T$ and a $\Sigma$-formula $\varphi$, a Satisfiability Modulo Theories(SMT) problem is the task of finding a model $M \in T$ such that $\models_M \varphi$.

\begin{lemma}\label{lemma:model-sat-unsat}
Given any $\Sigma$-model $M$ and $\Sigma$-formula $\varphi$, we have $\models_M \varphi $ if and only if $ \not\models_M \neg \varphi$
\end{lemma}

\begin{proof}
$\models_M \varphi \iff \varphi^M = 1 \iff 1 - \varphi^M = 0 \iff (\neg \varphi)^M = 0 \iff \not\models^M \neg \varphi$
\end{proof}

\begin{definition}
Let $T$ be a $\Sigma$-theory. A $\Sigma$-formula $\varphi$ is:
\begin{itemize}
\item[$\bullet$] \emph{satisfiable} in $T$ or T-SAT if and only if for all $M \in T$ we have $\models_{M} \varphi$,
\item[$\bullet$] \emph{valid} in $T$ or T-VALID if and only if for all $M \in T$ we have $\models_{M} \varphi$,
\item[$\bullet$] \emph{unsatisfiable} in $T$ or T-UNSAT if and only if for all $M \in T$ we have $\not\models_{M} \varphi$, and
\item[$\bullet$] \emph{unknown} in $T$ or T-UNKNOWN if and only if for all $M \in T$, $\varphi^M = \mathring{u}$.
\end{itemize}
\end{definition}

\begin{lemma} \label{lemma:theory-valid-unsat}
If $T$ be a $\Sigma$-theory, then $\varphi$ is T-VALID if and only if $\neg\varphi$ is T-UNSAT
\end{lemma}

\begin{proof}
$\varphi$ is T-VALID $\iff \forall M \in T; \; \models_{M} \varphi \iff \forall M \in T; \; \not\models_{M} \neg\varphi$ (Lemma \ref{lemma:model-sat-unsat}) $\iff \neg\varphi$ is T-UNSAT.
\end{proof}

\begin{lemma} \label{lemma:subtheory-SAT}
If $T' \subseteq T$, then $\varphi$ is $T'$-SAT implies that $\varphi$ is $T$-SAT.
\end{lemma}

\begin{proof}
$\varphi$ is $T'$-SAT $\implies \exists M \in T'; \; \models_M \varphi \implies \exists M \in T; \; \models_M \varphi$ (because $T' \subseteq T$) $\implies \varphi$ is $T$-SAT.
\end{proof}

\section{Polynomial Constraints over Real Numbers}
\subsection{Syntax}
We instantiate the signature $\Sigma^p = (S^p, P^p, F^p, V^p, \alpha^p)$ in Section ~\ref{subsection:smt-syntax} for polynomial constraints as following:
\begin{enumerate}
\item $S^p = \{Real\}$
\item $P^p = \{\succ, \prec, \succeq, \preceq, \approx, \not\approx\}$
\item $F^p = \{\oplus, \ominus, \otimes, \mathbf{1}\}$
\item for all $p \in P^p$, $\alpha^p(p) = (Real, Real)$
\item for all $f \in F^p\setminus \{\mathbf{1}\}$, $\alpha^p(f) = (Real, Real, Real)$ and $\alpha^p(\mathbf{1})=Real$
\item for all $v \in V, \; \alpha^p(v) = Real$
\end{enumerate}
A polynomial and a polynomial constraint are a $\Sigma^p$-term (we referred as letters $f$ or $g$) and a $\Sigma^p$-formula respectively. 


\begin{definition}
Given a polynomial $f$, the set of its variables which is denoted as $var(f)$ is defined recursively as following:
\begin{enumerate}
\item $var(v) = \{v\}$ for $v \in V$.
\item $var(\mathbf{1}) = \emptyset$.
\item $var(f_1 \circ f_2) = var(f_1) \cap var(f_2)$ with $\circ \in \{\oplus, \ominus, \otimes\}$.
\end{enumerate}
\end{definition}

\subsection{Semantics}
A model $M^p_{\mathbb{R}} = (\mathbb{R}, I^p_{\mathbb{R}})$ over real numbers for polynomial constraints contains the set of reals number $\mathbb{R}$ and a map $I$ that satisfies the following properties.
\begin{enumerate}
\item $I^p_{\mathbb{R}}(Real) = \mathbb{R}$.
\item $\forall p \in P$; $I^p_{\mathbb{R}}(p)$ is a function from $\mathbb{R} \times \mathbb{R}$ to $\{1, 0\}$ such that \[I^p_\mathbb{R}(p)(r_1, r_2) = \left\{ 
  \begin{array}{c l}
    1 & \quad \text{if } r1 \; p_{\mathbb{R}} \; r2\\
    0 & \quad \text{otherwise }\\
  \end{array} \right.\] where $(\succ_\mathbb{R}, \prec_{\mathbb{R}},\succeq_\mathbb{R},\preceq_{\mathbb{R}}, \approx_\mathbb{R}, \not\approx_\mathbb{R}) = (>, <, \ge, \le, =, \neq)$.
\item $\forall f \in F \setminus \{\mathbf{1}\}$; $I^p_{\mathbb{R}}(f)$ is a function from $\mathbb{R} \times \mathbb{R}$ to $\mathbb{R}$ such that \[I^p_{\mathbb{R}}(f)(r_1, r_2)  = r_1 \; f_{\mathbb{R}} \; r_2\] where $(\oplus_{\mathbb{R}}, \ominus_{\mathbb{R}}, \otimes_{\mathbb{R}}) = (+, -, *)$.
\item $I^p_\mathbb{R}(\mathbf{1}) = 1$
\item $\forall v \in V$; $I^p_{\mathbb{R}}(v) \in \mathbb{R}$.
\end{enumerate}
The valuation of polynomials ($\Sigma^p$-terms) and polynomial constraints ($\Sigma^p$-formulas) against a model $M^p_\mathbb{R}$ follows Definition~\ref{def:smt-valuation}. 

The theory of real numbers is $T^p_{\mathbb{R}} = \{M^p_{\mathbb{R}} | M^p_{\mathbb{R}}$ is a model of real numbers $\}$.
By this instantiation, each model differs to another by the mapping from variables to real numbers. As a result, an assignment of real numbers to variables, e.g. $\{v \mapsto r \in \mathbb{R} | v \in V\}$, can be used to represent a model. Given a map $\theta = \{v \mapsto r \in \mathbb{R}| v \in V \}$, $\theta^p_\mathbb{R}$ denotes the model represented by $\theta$.

Moreover, because all the predicate and function symbols' interpretations are total functions, so a given polynomial constraint $\varphi$ cannot be $T^p_\mathbb{R}$-UNKNOWN. In the other words, $\varphi$ can only be $T^p_\mathbb{R}$-SAT, $T^p_\mathbb{R}$-VALID, or $T^p_\mathbb{R}$-UNSAT.

From now on, we focus on $\Sigma^p$-formulas $\varphi$ of the forms:
\begin{center}
\begin{tabular}{l r l l}
$\varphi$ &::=& $p(f_1, f_2)$ & where $p \in P$\\ 
&$\mid$ & $\varphi_1 \wedge \varphi_2$& \\
\end{tabular}
\end{center}
because this does not lose the generality. Given a general polynomial constraint $\varphi$ that is formed by the syntax in Section~\ref{subsection:smt-syntax}:
\begin{itemize}
\item[$\bullet$] If we consider each formula $p(f_1, f_2)$ as an propositional symbols, we can first convert $\varphi$ into an CNF formula and then use DPLL procedure to infer a sequence of literals that satisfies $\varphi$ (in terms of propositional logic).  
\item[$\bullet$] The sequence of literals may contains some literals of the form $\neg p(f_1, f_2)$. However, from the semantics of polynomial constraints, we can change:
\begin{center}
\begin{tabular}{r c l}
$\neg(\succ (f_1, f_2))$ & to & $\preceq (f_1, f_2)$ \\
$\neg(\prec (f_1, f_2))$ & to & $\succeq (f_1, f_2)$ \\
$\neg(\succeq (f_1, f_2))$ & to & $\prec (f_1, f_2)$ \\
$\neg(\preceq (f_1, f_2))$ & to & $\succ (f_1, f_2)$ \\
$\neg(\approx (f_1, f_2))$ & to & $\not\approx (f_1, f_2)$ \\
$\neg(\not\approx (f_1, f_2))$ & to & $\approx (f_1, f_2)$ \\
\end{tabular}
\end{center}
\item The remaining task is solving the SMT problem with the constraint is the conjunction of literals in the sequence.
\end{itemize}

\subsubsection*{Representing (sub-)theory of real numbers as a constraint of real intervals}
The signature of the first order logic is instantiated as $\Sigma^I = (S^I, P^I, F^I, \alpha^I)$ for real interval constraints such that
\begin{enumerate}
\item $S^I = \{Real, Interval\}$,
\item $P^I = \{\in\}$,
\item $F^I = \{c \mid c \text{ is a constant}\}$,
\item $\alpha^I(\in) = (Real, Interval)$,
\item for all $c \in F^I, \; \alpha^I(c) = Interval$, and 
\item $\forall v \in V; \; \alpha^I(v) = Real$.
\end{enumerate}
We call $\Sigma^I$-formula is an interval constraint. The interval constraints in this thesis is represented by symbol $\Pi$ with possibly subscription.
A model $M^I_{\mathbb{R}} = (\mathbb{I} \cup \mathbb{R}, I^I_\mathbb{R})$ of real intervals consists of the union of real intervals $\mathbb{I}$ which is defined later in Definition~\ref{def:real_intervals} and real numbers $\mathbb{R}$ and a map $I^I_\mathbb{R}$ that satisfies the following properties.
\begin{enumerate}
\item $I^I_\mathbb{R}(Real) = \mathbb{R}$ and $I^I_\mathbb{R}(Interval) = \mathbb{I}$.
\item $I^I_\mathbb{R}(\in) = \mathbb{R} \times \mathbb{I} \mapsto \{1, 0\}$ such that \[I_\mathbb{R}^I(\in)(r, \langle a, b \rangle) = \left\{ 
  \begin{array}{c l}
    1 & \quad \text{if } a \le r \le b\\
    0 & \quad \text{otherwise }\\
  \end{array} \right.\]
\item For all $c \in F^I$, $I^I_\mathbb{R}(c) \in \mathbb{I}$.
\item For all $v \in V$, $I^I_\mathbb{R}(v) \in \mathbb{R}$.
\end{enumerate}
The valuation of $\Sigma^\mathbb{I}$-terms and $\Sigma^\mathbb{I}$-terms follows Definition~\ref{def:smt-valuation}. The theory of real intervals is $T^I_\mathbb{R} = \{M^I_\mathbb{R} | M^I_\mathbb{R}$ is a model of real intervals$\}$.
By this instantiation, each model differs to another by the mapping from variables to real numbers. As a result, an assignment of real numbers to variables, e.g. $\{v \mapsto r \in \mathbb{R} | v \in V\}$, can be used to represent a model. Given an assignment $\theta = \{v \mapsto r \in \mathbb{R} | v \in V\}$, we denote $\theta^I_\mathbb{R}$ as a model of real intervals represented by $\theta$. If $\Pi$ is a $\Sigma^\mathbb{I}$-formula, the notation ${\Pi^p_\mathbb{R} = \{\theta^p_\mathbb{R} \mid \theta = \{v \mapsto r \in \mathbb{R} \mid v \in V\} \text{ and } \models_{\theta^I_\mathbb{R}} \Pi \}}$ represents the (sub-)theory of real numbers that each of its model contain the assignment from real numbers to variables that (intuitively) satisfies $\Pi$.



% % % % % % % % % % % 