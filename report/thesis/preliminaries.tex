\chapter{Preliminaries}
\section{Abstract DPLL}
\subsection{Syntax}
\begin{definition}
The signature of propositional logic has an alphabet consisting of
\begin{enumerate}
\item a set of proposition symbols P and
\item logical connectives: $\wedge, \vee, \rightarrow, \leftrightarrow, \neg, \bot$.
\end{enumerate}
\end{definition}

\begin{definition}
The set PROP of propositions is the set that satisfies the following properties
\begin{enumerate}
\item $\forall i \in \mathbb{N} \; p_i \in PROP$,
\item $\bot \in PROP$,
\item $\varphi, \psi \in PROP \implies \varphi \circ \psi \in PROP$ where $\circ \in \{\wedge, \vee, \rightarrow, \leftrightarrow\}$,
\item $\varphi \in PROP \implies \neg\varphi \in PROP$.
\end{enumerate}
\end{definition}
\subsection{Semantics}
\begin{definition}
A model $M$ is a map $M : \{p_i | i \in \mathbb{N}\} \mapsto \{0, 1\}$
\end{definition}
\begin{definition} \label{def:pro-val}
Given a model $M$ and a proposition $\varphi \in PROP$, the valuation of $\varphi$ which is denoted by $\varphi^M$ is defined recursively as:
\begin{enumerate}
\item If $\varphi = p_i$ for some $i$, then $\varphi^M = M(p_i)$.
\item If $\varphi = \bot$, then $\varphi^M = 0$.
\item If $\varphi = \varphi_1 \wedge \varphi_2$, then $\varphi^M = min(\varphi_1^M,
 \varphi_2^M)$.
\item If $\varphi = \varphi_1 \vee \varphi_2$, then $\varphi^M = max(\varphi_1^M, \varphi_2^M)$.
\item If $\varphi = \varphi_1 \rightarrow \varphi_2$, then $\varphi^M = max(1 - \varphi_1^M, \varphi_2^M)$. 
\item If $\varphi = \varphi_1 \leftrightarrow \varphi_2$, then $\varphi^M = \left\{ 
  \begin{array}{c l}
    1 & \quad \text{if } \varphi_1^M = \varphi_2^M\\
    0 & \quad \text{otherwise }\\
  \end{array} \right.$
\item If $\varphi = \neg \varphi'$, then $\varphi^M = 1 - \varphi'^M$.
\end{enumerate}
\end{definition}


Given two propositions $\varphi_1$ and $\varphi_2$, we denote $\varphi_1 \models_{SAT} \varphi_2$ if and only if for all models M, $\varphi_1^M = 1$ implies $\varphi_2^M = 1$.
\section{Satisfiability Modulo Theories - SMT}
\subsection{Syntax} \label{subsection:smt-syntax}
\begin{definition}
A signature $\Sigma$ is a 4-tuple $(S, P, F, V, \alpha)$ containing a set $S$ of sorts, a set $P$ of predicate symbols, a set $F$ of function symbols, a set of variables, and a map $\alpha$ which associates symbols to their sorts.
\begin{itemize}
\item $\forall p \in P; \alpha(p)$ is a $n$-tuple argument sorts of $p$.
\item $\forall f \in F; \alpha(f)$ is a $n$-tuple of argument and returned sorts of $f$.
\item $\forall v \in V; \; \alpha(v)$ represents the sort of variable $v$.
\end{itemize}
\end{definition}
 
Followings are the definitions of terms and formulas.

\begin{definition}
The set $TERM$ of terms is the set that satisfies the properties
\begin{enumerate}
\item $v \in V \Rightarrow v \in TERM$
\item $t_1,\cdots,t_n \in TERM$; $f \in F \Rightarrow f(t_1,\cdots, t_n) \in TERM$
\end{enumerate}
\end{definition}

\begin{definition}
We only focus on quantifiers free formula, so the set $FORM$ of formulas is the set that satisfies the properties:
\begin{enumerate}
\item $\bot \in FORM$
\item $t_1,\cdots,t_n \in TERM$; $p \in P \Rightarrow p(t_1,\cdots, t_n) \in FORM$
\item $\varphi, \psi \in FORM \Rightarrow (\varphi \circ \psi) \in FORM$ where $\circ \in \{\wedge, \vee, \rightarrow, \leftrightarrow\}$
\item $\varphi \in FORM \Rightarrow \neg\varphi \in FORM$
\end{enumerate}
\end{definition}

\subsection{Semantics}

\begin{definition}
Let $\Sigma$ is a signature. A model $M$ of $\Sigma$ is a pair $(U, I)$ in which $U$ is the universe and $I$ is the interpretation of symbols.
\begin{itemize}
\item $\forall s \in S$; $I(s) \subseteq U$ specifies the possible values of sort $s$.
\item $\forall f \in F$; $I(f) = \{(t_1,\cdots, t_n)| t_1 \in I(s_1),\cdots, t_n \in I(s_{n-1})\} \mapsto I(s_n)$ with $, \alpha(f) = (s_1,\cdots, s_n)$
\item $\forall p \in P$; $I(p) = \{(t_1,\cdots, t_n)| t_1 \in I(s_1),\cdots, t_n \in I(s_n)\} \mapsto \{0, 1\}$ with $\alpha(p) = (s_1,\cdots, s_n)$
\item $\forall v \in V$; $I(v) \in I(\alpha(v))$
\end{itemize}
\end{definition}

The interpretation of one predicate symbol is allowed to be not total and the symbol $\mathring{u}$ (unknown) is used to indicate the result of undefined operations. For further convenience, we also define the following relations: $\mathring{u} < 0, 1$ and $\mathring{u} > 0, 1$ and the following arithmetic $1 - \mathring{u} = \mathring{u}$ which are useful when we evaluate the values of logical connectives, i.e. $\wedge$, $\vee$, $\rightarrow$, $\leftrightarrow$, or $\neg$. 

\begin{definition}
Let $\Sigma$ is a signature. A $\Sigma$-theory T is a (infinite) set of $\Sigma$-models.
\end{definition}
A theory $T'$ is a subset of theory $T$ iff $T' \subseteq T$.

\begin{definition}
Let $\Sigma = (S, P, F, \alpha)$, $t$, $\varphi$ and $M=(U, I)$ are a signature, a $\Sigma-$term, a $\Sigma-$formula and a $\Sigma-$model respectively. The valuations of $t$ against $M$ which is denoted by $t^M$ is defined recursively as:
\begin{enumerate}
\item If $t = v \in V$, then $t^M = I(v)$.
\item If $t = f(t_1, \cdots, t_n)$, then $t^M = \left\{ 
  \begin{array}{c l}
    I(f)(t_1^M, \cdots, t_n^M) & \quad \text{if } (t_1^M, \cdots, t_n^M) \in Dom(I(f))\\
    \mathring{u} & \quad \text{otherwise }\\
  \end{array} \right.$ for $f \in F$ and $t_1,\cdots, t_n \in TERM$
\end{enumerate}
Similarly, the valuation $\varphi^M$ of $\varphi$ is defined as:
\begin{enumerate}
\item If $\varphi = p(t_1,\cdots,t_n)$, then $\varphi^M = \left\{ 
  \begin{array}{c l}
    I(p)(t_1^M,\cdots,t_n^M) & \quad \text{if } (t_1^M, \cdots, t_n^M) \in Dom(I(p))\\
    \mathring{u} & \quad \text{otherwise }\\
  \end{array} \right.$ for $p \in P$ and $t_1,\cdots, t_n \in TERM$
\item If $\varphi = \bot, \neg\varphi', \varphi_1 \circ \varphi_2$ for $\circ \in \{\wedge, \vee, \rightarrow, \leftrightarrow\}$ and $\varphi', \varphi_1, \varphi_2 \in FORM$, then $\varphi^M$ is defined similarly as in Definition~\ref{def:pro-val}.
\end{enumerate}
We say that $M$ satisfies $\varphi$ which is denoted by $\models_M \varphi$ iff $\varphi^M = 1$. If $\varphi^M = 0$, $\not\models_M \varphi$ is used to denote that $M$ does not satisfy $\varphi$.
\end{definition}

\begin{lemma}\label{lemma:model-sat-unsat}
Given any $\Sigma$-model $M$ and $\Sigma$-formula $\varphi$, we have $\models_M \varphi \iff \not\models_M \neg \varphi$
\end{lemma}

\begin{proof}
$\models_M \varphi \iff \varphi^M = 1 \iff 1 - \varphi^M = 0 \iff (\neg \varphi)^M = 0 \iff \not\models^M \neg \varphi$
\end{proof}

\begin{definition}
Let $T$ be a $\Sigma$-theory. A $\Sigma$-formula $\varphi$ is:
\begin{itemize}
\item satisfiable in $T$ or T-SAT iff $\exists M \in T$; $\models_{M} \varphi$
\item valid in $T$ or T-VALID iff $\forall M \in T$; $\models_{M} \varphi$
\item unsatisfiable in $T$ or T-UNSAT iff $\forall M \in T$; $\not\models_{M} \varphi$
\item unknown in $T$ or T-UNKNOWN iff $\forall M \in T$; $\varphi^M = \mathring{u}$
\end{itemize}
\end{definition}

\begin{lemma} \label{lemma:theory-valid-unsat}
If $T$ be a $\Sigma$-theory, then $\varphi$ is T-VALID $\iff$ $\neg\varphi$ is T-UNSAT
\end{lemma}

\begin{proof}
$\varphi$ is T-VALID $\iff \forall M \in T; \; \models_{M} \varphi \iff \forall M \in T; \; \not\models_{M} \neg\varphi$ (Lemma \ref{lemma:model-sat-unsat}) $\iff \neg\varphi$ is T-UNSAT.
\end{proof}

\begin{lemma} \label{lemma:subtheory-SAT}
If $T' \subseteq T$, then $\varphi$ is $T'$-SAT $\implies \varphi$ is $T$-SAT.
\end{lemma}

\begin{proof}
$\varphi$ is $T'$-SAT $\implies \exists M \in T'; \; \models_M \varphi \implies \exists M \in T; \; \models_M \varphi$ (because $T' \subseteq T$) $\implies \varphi$ is $T$-SAT.
\end{proof}

\section{Polynomial Constraints over Real Numbers}
\subsection{Syntax}
We instantiate the signature $\Sigma^p = (S^p, P^p, F^p, V^p, \alpha^p)$ in Section ~\ref{subsection:smt-syntax} for polynomial constraints as following:
\begin{enumerate}
\item $S^p = \{Real\}$
\item $P^p = \{\succ, \prec, \succeq, \preceq, \approx, \not\approx\}$
\item $F^p = \{\oplus, \ominus, \otimes, \mathbf{1}\}$
\item $\forall p \in P^p$; $\alpha^p(p) = (Real, Real)$
\item $\forall f \in F^p\setminus \{\mathbf{1}\}$; $\alpha^p(f) = (Real, Real, Real)$ and $\alpha^p(\mathbf{1})=Real$
\item $\forall v \in V; \; \alpha^P(v) = Real$
\end{enumerate}
A polynomial and a polynomial constraint are a $\Sigma^p$-term and a $\Sigma^p$-formula respectively. However, for simplicity, we currently focus on small portion of $\Sigma^p$-formulas

\begin{definition}
Given a polynomial $f$, the set of its variables which is denoted as $var(f)$ is defined recursively as following:
\begin{enumerate}
\item $var(v) = \{v\}$ for $v \in V$.
\item $var(\mathbf{1}) = \emptyset$.
\item $var(f_1 \circ f_2) = var(f_1) \cap var(f_2)$ with $\circ \in \{\oplus, \ominus, \otimes\}$.
\end{enumerate}
\end{definition}

\subsection{Semantics}
A model $M^p_{\mathbb{R}} = (\mathbb{R}, I^p_{\mathbb{R}})$ over real numbers contains the set of reals number $\mathbb{R}$ and a map $I$ that satisfies the following properties.
\begin{enumerate}
\item $I^p_{\mathbb{R}}(Real) = \mathbb{R}$.
\item $\forall p \in P^p$; $I_{\mathbb{R}}(p) = \mathbb{R} \times \mathbb{R} \mapsto \{1, 0\}$ such that $ I_\mathbb{R}^p(p)(r_1, r_2) = \left\{ 
  \begin{array}{c l}
    1 & \quad \text{if } r1 \; p_{\mathbb{R}} \; r2\\
    0 & \quad \text{otherwise }\\
  \end{array} \right.$ where $(\succ_\mathbb{R}, \prec_{\mathbb{R}},\succeq_\mathbb{R},\preceq_{\mathbb{R}}, \approx_\mathbb{R}, \not\approx_\mathbb{R}) = (>, <, \ge, \le, =, \neq)$.
\item $\forall f \in F^p \setminus \{\mathbf{1}\}$; $I^p_{\mathbb{R}}(f) = \mathbb{R} \times \mathbb{R} \mapsto \mathbb{R}$ such that $I^p_{\mathbb{R}}(f)(r_1, r_2)  = r_1 \; f_{\mathbb{R}} \; r_2$ where $(\oplus_{\mathbb{R}}, \ominus_{\mathbb{R}}, \otimes_{\mathbb{R}}) = (+, -, *)$.
\item $I^p_\mathbb{R}(\mathbf{1}) = 1$
\item $\forall v \in V$; $I^p_{\mathbb{R}}(v) \in \mathbb{R}$.
\end{enumerate}
The theory of real numbers is $T^p_{\mathbb{R}} = \{M^p_{\mathbb{R}} | M^p_{\mathbb{R}}$ is a model of real numbers $\}$.
By this instantiation, each model differs to another by the mapping from variables to real numbers. As a result, an assignment of real numbers to variables can be used to represent a model $M^p_{\mathbb{R}}$; i.e. $\{v \mapsto r \in \mathbb{R} | v \in V\}$ represents a model. Given a map $\theta = \{v \mapsto r \in \mathbb{R}| v \in V \}$, $\theta^p_\mathbb{R}$ denotes the model represented by $\theta$.

\subsubsection*{Representing (sub-)theory of real numbers as a constraint of intervals}
The signature of the first order logic is instantiated as $\Sigma^I = (S^I, P^I, F^I, \alpha^I)$ for interval constraints:
\begin{enumerate}
\item $S^I = \{Real, Interval\}$
\item $P^I = \{\in\}$
\item $F^I = \{c | c \text{ is a constant}\}$
\item $\alpha^I(\in) = (Real, Interval)$ and $\forall c \in F^I; \; \alpha^I(c) = Interval$
\item $\forall c \in F^I; \; \alpha^I(c) = Interval$.
\item $\forall v \in V; \; \alpha^I(v) = Real$.
\end{enumerate}
We call $\Sigma^I$-formula is an interval constraint.
A model $M^I_{\mathbb{R}} = (\mathbb{I} \cup \mathbb{R}, I^I)$ over intervals contains the set of real numbers and real intervals $\mathbb{I} \cup \mathbb{R}$ and a map $I^I$ that satisfies the following properties.
\begin{enumerate}
\item $I^I(Real) = \mathbb{R}$ and $I^I(Interval) = \mathbb{I}$. $\mathbb{I}$ is the set of all real intervals which is defined later in Definition~\ref{def:real_intervals}.
\item $I^I(\in) = \mathbb{R} \times \mathbb{I} \mapsto \{1, 0\}$ such that $ I_\mathbb{R}^p(p)(r, \langle a, b \rangle) = \left\{ 
  \begin{array}{c l}
    1 & \quad \text{if } a \le r \le b\\
    0 & \quad \text{otherwise }\\
  \end{array} \right.$.
\item $\forall c \in F^I$; $I^I(c) \in \mathbb{I}$.
\item $\forall v \in V$; $I^I(v) \in \mathbb{R}$.
\end{enumerate}
The theory of real intervals is $T^I = \{M^I | M^I$ is a model of real intervals $\}$.
By this instantiation, each model differs to another by the mapping from variables to real numbers. As a result, an assignment of real numbers to variables can be used to represent a model $M^I$; i.e. $\{v \mapsto r \in \mathbb{R} | v \in V\}$ represents a model. Given a map $\theta = \{v \mapsto r \in \mathbb{R} | v \in V\}$, we denote $\theta^I$ as a model of real intervals. If $\Pi$ is an interval constraint, the notation $\Pi^p_\mathbb{R} = \{\theta^p_\mathbb{R} | \theta = \{v \mapsto r \in \mathbb{R} | v \in V\} \text{ and } \models_{\theta^I} \Pi \}$ represents the (sub-)theory of real numbers that each of its model (after converted into the model of intervals) satisfies the constraint $\Pi$.



% % % % % % % % % % % 