\chapter{Conclusion} \label{chap:conclusion}


This thesis presented improvement and extensions for an SMT solved {\bf raSAT} including heuristics to deal with exponential exploration of boxes, extensions for handling equations and handling constraints over integer numbers. From the experiments on standard SMT-LIB benchmarks, \textbf{raSAT} is able to solve large constraints (in terms of the number of variables) which are difficult for other tools.
The contributions of this work are as follows:
\begin{enumerate}
\item To deal with exponential growth of the number of boxes during refinement (interval decomposition), two strategies for \emph{selecting one variable} to decomposed and \emph{selecting one box} were proposed:
\begin{itemize}
\item[$\bullet$] \textbf{Selecting one box.} The box with more possiblity to satisfy the constraint is selected to explore, which is estimated by 
several heuristic measures, called {\em SAT likelyhood}, 
and \emph{the number of unsolved polynomial inequalities}.
\item[$\bullet$] \textbf{Selecting one variable.} The most influential variable is selected for multiple test cases and decomposition. 
This is estimated by {\em sensitivity} which is determined during the approximation process.
\end{itemize}
\item Two schemes of \emph{incremental search} are proposed for enhancing solving process: 
\begin{itemize} 
\item[$\bullet$] {\bf Incremental deepening}. 
raSAT follows the depth-first-search manner. In order to escape local exhaustive search, it starts searching with a threshold that each interval will be decomposed no smaller than it. 
If neither satisfiability nor unsatisfiability is detected, a smaller threshold is taken and raSAT restarts. 
\item[$\bullet$] {\bf Incremental widening}. 
Starting with a small intervals, if \textbf{raSAT} detects UNSAT, it enlarges input intervals and restarts. This strategy is effective in detecting satisfiability of constraints because small intervals reduce the number of boxes after decomposition.
\end{itemize}
\item \emph{Satisfiability confirmation} step by an error-bound guaranteed floating point package {\bf iRRAM}\footnote{% 
\tt http://irram.uni-trier.de}, to avoid soundess bugs caused by roundoff errors.
\item This work also implemented the idea of using Intermediate Value Theorem to show \emph{the satisfiability of multiple equations} which was suggested in \cite{VanKhanh201227}.
\item \textbf{raSAT} is also extended to \emph{handle constraints over integer numbers} by simple extension in the approximation process.
\end{enumerate}

\section*{Future Directions}
A number of ideas for the future works are:
\begin{enumerate}
\item \textbf{UNSAT core}: Two strategies \emph{SAT likelihood} and \emph{the number of unsolved inequalities} aim at boosting satisfiability detection. For unsatisfiable constraints, \emph{UNSAT core} is the key in expeditious detection. Although we have some ideas for this but they did not show much improvement in experiments. As a future work, more investigation is needed for \emph{UNSAT core}.
\item \textbf{Test case generation:} Currently \textbf{raSAT} randomly generates test cases in testing phase. We had the idea of using sensitivity to guide testing but this needs to be investigated in detail.
\item \textbf{Box Decomposition:} The key in UNSAT detection is also how to early isolate unsatisfiable intervals that is done through decomposition. Note that at the moment the decomposition strategy of \emph{raSAT} is taking its middle point.
\end{enumerate}