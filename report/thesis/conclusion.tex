\chapter{Conclusion}


This paper presented {\bf raSAT} loop, which mutually refines over and under--approximation 
theories. For polynomial inequlaity, we adopted interval arithemtic and testing for 
over and under-approximation theories, respectively. 
{\bf raSAT} loop is implemented as an SMT {\bf raSAT}. 
The result of Experiments on QF\_NRA in SMT-lib is encouraging, and 
{\bf raSAT} shows comparable and sometimes outperforming to existing SMTs, e.g., 
Z3 4.3, HySAT, and dReal. For instance, ****** which has ** variables and degree ** 
was solved by {\bf raSAT}, whereas none of above mentioned SMTS can. 
%

\section{Observation and Discussion} 

From experimental results in Section~\ref{sec:experiments} and~\ref{sec:expsmtlib}, 
we observe the followings. 
\begin{itemize}
\item The degree of polynomials will not affect much. 
\item The number of variables are matters, but also for Z3 4.3. 
The experimental results do not show exponential growth, and we expect 
the strategy of selection of an API in which related intervals are decomposed
seems effective in practice. By observing Zankl examples, we think the maximum 
number of variables of each API seems a dominant factor. 
\item Effects of the number of APIs are not clear at the moment. In simple benchmarks, 
{\bf raSAT} is faster than Z3 4.3, however we admit that we have set small degree $n=6$
for each API. 
\end{itemize}

For instance, {\em matrix-2-all-5,8,11,12} in Zankl 
contain a long monomial (e.g., $60$) with the max degree $6$, and 
relatively many variables (e.g., $14$), which cannot be solved by Z3 4.3, but 
{\bf raSAT} does. 
As a general feeling, if an API contains more than $30 \sim 40$ variables, 
{\bf raSAT} seems saturating. 
We expect that, adding to a strategy to select an API (Section~\ref{sec:intervaldecomp}), 
we need a strategy to select variables in the focus. We expect this can be designed 
with sensitivity (Example~\ref{examp:sensitivity}) and would work in practice. 
Note that sensitivity can be used only with noise symbols in Affine intervals. 
Thus, iSAT and RSOLVER cannot use this strategy, though they are based on IA, too. 

\section{Future Work}
%
{\bf raSAT} still remains in naive proto-type status, and 
there are lots of future work. 

\medskip \noindent 
{\bf UNSAT core}. 
\mizuhito{to be filled}

\medskip \noindent 
{\bf Exact confirmation}.
Currently, {\bf raSAT} uses floating point arithmetic. Thus, results can be unsound. 
We are planning to add a confirmation phase to confirm whether an SAT instance is exact
by roundoff error bound guaranteed floating arithmetic libraries, such as ****. 

\medskip \noindent 
{\bf Equality handling}. 
%Section~\ref{sec:eq} shows single equality handling. 
We are planning to extend the use of Intermediate Value Theorem to multiple equality with 
shared variables. 

\medskip \noindent 
{\bf Further strategy refiment}. 
\mizuhito{Test data generation, blanced box decomposition}


%%%
\suppress{
\subsubsection{Extension of {\bf raSAT} loop}
\begin{itemize}
\item {\bf Equality handling}: currently, {\bf raSAT} loop can handle only inequalities. 
Before applying ideal based technique, such as {\em Gr{\"o}bner basis}, 
we are planning to implement a non-constructive detection of equality 
by {\em intermediate value theorem}. 

\suppress{
\item{\textbf{Polynomial equality by Intermediate value theorem}:} 
Consider 
\begin{center}
$(x_1 \in (a_1,b_1) \wedge \cdots \wedge x_n \in (a_n,b_n))~\bigwedge 
\limits_{j}^m f_j(x_1,\cdots,x_n) > 0~\wedge~g(x_1,\cdots, x_n) = 0.$
\end{center}
SAT can be proved by two steps. First, find a box of the product of 
$(l_{ik},h_{ik}) \subseteq (l_i,h_i)$ (by interval arithmetic) such that 
\begin{center}
$\forall x_1 \in (l_{1k},h_{1k}) \cdots x_n \in (l_{nk},h_{nk}).~\bigwedge 
 \limits_{j}^m f_j(x_1,\cdots,x_n) > 0$~~~~(IA-VALID) 
\end{center}
and find two instances (by testing) in the box with $g(a_1,\cdots,a_n) > 0$ 
and $g(b_1,\cdots,b_n) < 0$. By Intermediate value theorem, we can conclude 
$\exists x_1 \in (l_1,h_1) \cdots x_n \in (l_n,h_n).~g(x_1,\cdots, x_n) = 0$. 
}

\item \textbf{Solving polynomial constraints on integers}: 
In integer domain, the number of test data is finite if interval constraints are bounded. 
Then, Test-UNSAT implies UNSAT if all possible test data are generated. 
A tight interaction between testing and interval decomposition could be investigated.
Mixed integers are also challenging. 
\end{itemize}


\subsubsection{\textbf{raSAT} Development}

\begin{itemize}
\item \textbf{Avoiding local optimal}: 
we borrow an idea of \emph{restart} in MiniSAT for escaping from hopeless local search 
(i.e., solution set is not dense or empty). 
\emph{Heuristics} would be, after a deep interval decomposition of 
a box and Test-UNSAT are reported, backtrack occurs to choose a randomly selected box. 

\item \textbf{Separation of linear constraints}: 
Many benchmarks contain linear constraints. Current implementation does not have 
any tuning, but {\bf raSAT} loop only. 
Practically, separating linear and non-linear constraints and solving them 
in a coordinated way between Presbuger arithmetic and {\bf raSAT} would improve. 
During this separation, variables of intersecting linear constraints would be candidates 
for interval decompositions. 

\item \textbf{Incremental DPLL}: For interactions with the SAT solver, 
we currently apply the very lazy theory learning. Combination with 
\emph{eager} theory propagation would improve, in which we can propagate 
a conflict from a partial truth assignment instead of waiting 
for a full truth assignment obtained by SAT solver.
\end{itemize}
}
%%

%%
\suppress{
\subsection{Applications}
\begin{itemize}
\item \textbf{Checking overflow and roundoff error}: In the computers, the real numbers are represented by finite numbers (i.e., floating point numbers, fixed point numbers). Due to finite representation, the over-flow and roundoff errors (OREs) \cite{ngocsefm, ngocase} may occur. The OREs will be propagated through computations of the program. Further, the computations themselves also cause OREs because the arithmetic needs to round the result to fit the number format. Besides, OREs are also affected by types of statements, i.e., branch, loop, assignment statements.
By symbolic execution, ORE constraints are propagated from a program and ORE problems are reduced to problems of solving ORE constraints for verifying whether OREs occur. 
%For solving ORE constraint, combination of the new form of affine arithmetic ($CAI_1$) and \emph{sensitivity} (i.e., high degrees, )

\item \textbf{Loop invariant generation}: The problem of linear invariant generation is often reduced to the problem of non-linear constraint solving. 
Since Farkas's Lemma \cite{Colon03} uses the product of matrices with polynomial constraint solving, we can extend the target for non-linear invariant generation.
%Based on Farkas' Lemma \cite{Colon03}, non-linear constraints on coefficients of the target linear invariant are generated and a satisfiable instance of these constraints is a candidate of the linear invariant. 
\end{itemize}
}
%%