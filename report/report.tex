\documentclass[12pt]{article}

\title{raSAT - report}


\begin{document}
\maketitle
\section{Old status}

In the experiments, we allow $2^{10}$ test cases, only one variable is decomposed. Two experiments:

\subsection{No sbox}
We continue decompose intervals even when their length is very small.

Result(see "1.1.xls" file): 50 problems in Zankl, round-off error is more likely to exist.

\subsection{SAT directed, sbox=0.1}
After one interval is decomposed, we use IA to evaluate the two new intervals. We choose the interval which makes the TEST-UNSAT API have a longer SAT.

Result ("1.2.xls" file): 42 problems in Zankl.

\section{Current status}

\subsection{Unbalanced decomposition using sensitivity}
\label{subsec:unbalanced}
Example: Suppose we have the constraint: 

$f = -x_{15}*x_8+x_{15}*x_2-x_{10}*x_{16}>0$. 

With $x_2 \in [9.9, 10], x_8 \in [0, 0.1], x_{10} \in [0, 0.1], x_{15} \in [0, 10], x_{16} \in [0, 10]$, the result of AF2 for $f$ is: $0.25 \epsilon_2 - 0.25 \epsilon_8 - 0.25 \epsilon_{10} + 49.5\epsilon_{15} - 0.25\epsilon_{16} + 0.75\epsilon_{+-} + 49.25$. The estimated bound for $f$ is thus $[-2, 100.5]$. We will chose $x_{15} \in [0, 10]$ for decomposition based on sensitivity. In addition, the coefficient of $\epsilon_{15}$ is positive ($49.5$), we can conclude that if $\epsilon_{15}$ increase, the value of $f$ will likely increase. Because $x_{15} = 5+5\epsilon_{15}$, $\epsilon_{15}$ increases when $x_{15}$ increases. As the consequence, if $x_{15}$ increases, $f$ will likely increases. Besides, because the constraint is $f>0$, we expect the intervals that make the value of $f$ as high as possible. So we will decompose $x_{15} \in [0, 10]$ into $x_{15} \in [0, 9.9] \lor x_{15} \in [9.9, 10]$, for example; and we will force $x_{15} \in [9.9, 10]$ to be selected next by MiniSAT. That means the next considered intervals are $x_2 \in [9.9, 10], x_8 \in [0, 0.1], x_{10} \in [0, 0.1], x_{15} \in [9.9, 10], x_{16} \in [0, 10]$. The estimated bounds for $f$ is $[96.01, 100.5]$.

This is a comparison between the proposed unbalanced decomposition and balanced decomposition. Suppose the constraints is still the same $f = -x_{15}*x_8+x_{15}*x_2-x_{10}*x_{16}>0$. The intervals for variables are all $[0, 10]$, that means $x_2 \in [0, 10], x_8 \in [0, 10], x_{10} \in [0, 10], x_{15} \in [0, 10], x_{16} \in [0, 10]$. We will compare methods in terms of the number of steps they need to find IA-VALID intervals of $f>0$.
\begin{center}
    \begin{tabular}{ | l | l | l | }
    \hline
    Step & Balanced Decomposition & Unbalanced Decompostion\\ \hline
    1 & [-200, 150] & [-200, 150] \\ \hline
    2 &  [-175, 150] & [-150.5, 150] \\ \hline
    3 & [-125, 125] & [-51.5, 100.5] \\ \hline
    4 &  [-100, 125] & [-2. 100.5] \\ \hline
    5 &  [-75, 125] & [96.01 100.5] - IA-VALID \\ \hline 
    6 &  [-56.25, 125] &  \\ \hline
    7 &  [-37.5, 125] & \\ \hline
    8 &  [-18.75, 125] & \\ \hline
    9 &  [6.25, 112.5] - IA-VALID &  \\ \hline
    \end{tabular}
\end{center}

\subsection{Test cases based on Sensitivity}
\label{subsec:testBasedSen}
The signs of the coefficients of noise errors can also guide the testing phase. Let's consider the above example. The constraint is $f = -x_{15}*x_8+x_{15}*x_2-x_{10}*x_{16}>0$. With $x_2 \in [9.9, 10], x_8 \in [0, 0.1], x_{10} \in [0, 0.1], x_{15} \in [0, 10], x_{16} \in [0, 10]$, the result of AF2 for $f$ is: $0.25 \epsilon_2 - 0.25 \epsilon_8 - 0.25 \epsilon_{10} + 49.5\epsilon_{15} - 0.25\epsilon_{16} + 0.75\epsilon_{+-} + 49.25$. The coefficient of $\epsilon_2$ is positive ($0.25$), then we expect the test case of $x_2$ is as high as possible in order to satisfy $f>0$. We will thus take the upper bound value of $x_2$, i.e. $10$. Similarly, we take the test cases for other variables: $x_8=0, x_{10}=0, x_{15}=10, x_{16}=0$. With these test cases, we will have $f=100 > 0$.

\subsection{Experiments}
In the experiments, we set $sbox=0.1$. Unbalanced decomposition uses sbox, for example $x_{15} \in [0, 10]$ into $x_{15} \in [0, 10-sbox] \lor x_{15} \in [10-sbox, 10]$. Time out is $500$s. There are 4 experiments which are different in the way of generating test cases. All the experiments use the unbalanced decomposition as described in \ref{subsec:unbalanced}. 

\subsubsection{All test cases are random}
The number of test cases is $2^{10}$.
All the value of variables are randomly generated.
Result ("2.3.1.xls" file): 44 problems solved

\subsubsection{1 test case based on sensitivity}
The number of test cases is only 1. Each variable is assigned one value based on its sensitivity (section \ref{subsec:testBasedSen}).
Result ("2.3.2.xls" file): 48 problems solved.

\subsubsection{At least each variable has one random value}
The number of test cases is $2^{10}$. 
\begin{enumerate}
  \item First 10 variables: Each variable will have:
  \begin{itemize}
    \item 1 random value, and
    \item another value based on sensitivity (section \ref{subsec:testBasedSen}).
  \end{itemize}
  \item Other variables: Each variable has one random value as test case.
\end{enumerate}
Result ("2.3.3.xls" file): 47 problems solved.

\subsubsection{First 10 variables has one random value}
The number of test cases is $2^{10}$. 
\begin{enumerate}
  \item First 10 variables: Each variable will have:
  \begin{itemize}
    \item 1 random value, and
    \item another value based on sensitivity (section \ref{subsec:testBasedSen}).
  \end{itemize}
  \item Other variables: Each variable has one value based on sensitivity (section \ref{subsec:testBasedSen}).
\end{enumerate}
Result ("2.3.4.xls" file): 53 problems solved.
\end{document}